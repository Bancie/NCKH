\documentclass{beamer}
\mode<presentation>
\setbeamertemplate{bibliography item}{}
\usepackage[utf8]{vietnam}
\usepackage{beamerthemesplit}
\usepackage{graphicx}
\usepackage{booktabs}
\usepackage{amsmath}
\usepackage{textpos}
\usepackage{pgfplots}
\usepackage{tikz}
\usepackage{hyperref}
\usepackage{caption}
\usetikzlibrary{shapes.geometric, arrows}
\usetikzlibrary {datavisualization} 
\pgfplotsset{compat=1.18, width = 7cm}
\usetikzlibrary{patterns}
\setbeamertemplate{bibliography item}[text]
\usetheme{Ilmenau} % AnnArbor, Ilmenau, Darmstadt, Dresden, CambridgeUS, Frankfurt, Singapore
\newtheorem{dn}{Định nghĩa}[section]
\newtheorem{dl}{Định lý}[section]
\newtheorem{tc}{Tính chất}[section]
\newtheorem{hq}{Hệ quả}[section]
\newtheorem{bd}{Bổ đề}[section]
\newtheorem{md}{Mệnh đề}[section]
\newtheorem{vd}{Ví dụ}[section]
\newtheorem{nx}{Nhận xét}[section]
\newcommand{\dom}{\text{{\rm dom}}}
\newcommand{\epi}{\text{{\rm epi}}}
\newcommand{\Min}{\text{{\rm Min}}}
\setbeamertemplate{theorems}[numbered]
\setbeamertemplate{definitions}[numbered]
\setbeamertemplate{footline}[frame number]
\usepackage{algorithm}
\usepackage{color}
\usepackage{algorithmic}
\usepackage{footmisc}
\usepackage{indentfirst} 
\usepackage{comment}
\AtBeginEnvironment{proof}{%
  \setbeamercolor{block title}{use=example text,fg=white,bg=example text.fg!75!black}
  \setbeamercolor{block body}{parent=normal text,use=block title example,bg=block title example.bg!10!bg}
}
\renewcommand{\thefootnote}{\arabic{footnote}}
\usefonttheme{professionalfonts}
\setbeamercolor{normal text}{bg=white,fg=black}
\renewcommand{\thefootnote}{\arabic{footnote}}
\beamertemplatetransparentcoveredhigh
\title[]{\fontsize{13pt}{10pt}\selectfont {\bf \LARGE Phương pháp nhánh cận}\\}
\author[]{\bf Nguyễn Chí Bằng \\}
\small{\date{\today}}
\begin{document}
\begin{frame}
\small
\setlength{\parindent}{4em}
\noindent \textbf{Bước 1. Thiết lập} \\
Đặt $\mathcal{L}:=\{N_0 \}$, $z^*=z_p$ và $(x^*,y^*)=(x,y)$. \\
\noindent \textbf{Bước 2. Kiểm tra} \\
Nếu $\mathcal{L} = \emptyset$ thì nghiệm tối ưu của bài toán là $(x^*,y^*)$ và giá trị \\ tối ưu là $z^*$ và bài toán được giải. \\
Nếu $\mathcal{L} \neq \emptyset$, chuyển sang bước 3. \\ 
\noindent \textbf{Bước 3. Chọn nút} \\
Chọn nút $N_i$ từ danh sách $\mathcal{L}$ và xoá khỏi $\mathcal{L}$ sau đó chuyển sang \\ bước 4. \\ 
\noindent \textbf{Bước 4. Xác định cận} \\ 
Giải bài toán $(P_i)$, nếu bài toán vô nghiệm, quay lại bước 2, \\ nếu không, chuyển sang bước 5. \\ 
\noindent \textbf{Bước 5. Gọt nghiệm} \\
Nếu tồn tại $x^i \notin Z^n_+$, ta thêm nút $N_{i1}, \ldots , N_{ik}$ vào $\mathcal{L}$ và quay \\ về bước 2. \\
Nếu không tồn tại $x^i \notin Z^n_+$, tức $\forall x^i \in Z^n_+$, ta đặt $z_i = z^*$, \\ $(x^i,y^i) = (x,y)$ và quay lại bước 2.
\end{frame}    
\begin{frame}{Sơ đồ thuật giải}
  \center
  \begin{figure}
  \begin{tikzpicture}[scale=0.33, every node/.style={scale=0.33}, node distance=4cm]
      % Define styles
      \tikzstyle{startstop} = [rectangle, rounded corners, minimum width=1cm, minimum height=1cm, align=center, draw=black, fill=red!30]
      \tikzstyle{process} = [rectangle, minimum width=1cm, minimum height=1cm, align=center, draw=black, fill=orange!30]
      \tikzstyle{decision} = [diamond, aspect=2, minimum width=1cm, minimum height=1cm, align=center, draw=black, fill=green!30]
      \tikzstyle{io} = [trapezium, trapezium left angle=70, trapezium right angle=110, minimum width=1cm, minimum height=1cm, align=center, draw=black, fill=blue!30]
      \tikzstyle{arrow} = [thick,->,>=stealth,line width=0.5pt]
      % Place nodes
      \node (start) [startstop, xshift=-5cm] {Bắt đầu};
      \node (b0) [io, right of=start] {$\mathcal{L}:=\{N_0 \}$ \\ $z^*=z_p$ \\ $(x^*,y^*)=(x,y)$};
      \node (b1) [decision, right of=b0] {$\mathcal{L} = \emptyset$?};
      \node (b2) [process, right of=b1] {Chọn $N_i$ từ $\mathcal{L}$ \\ và xoá khỏi $\mathcal{L}$};
      \node (b3) [process, right of=b2] {Giải $(P_i)$};
      \node (b4) [decision, right of=b3] {Vô nghiệm?};
      \node (b5) [decision, right of=b4] {$\exists x^i \notin Z^n_+$?};
      \node (b6) [io, above of=b5] {$z_i = z^*$, $(x^i,y^i) = (x,y)$};
      \node (b7) [io, right of=b5] {Thêm \\ $N_{i1}, \ldots , N_{ik}$ \\ vào $\mathcal{L}$};
      \node (kl) [process, below of=b1] {bài toán \\ được giải};
      \node (stop) [startstop, right of=b7] {Kết thúc};
      % Draw arrows
      \draw [arrow] (start) -- (b0);
      \draw [arrow] (b0) -- (b1);
      \draw [arrow] (b1) -- node[anchor=south] {no} (b2);
      \draw [arrow] (b2) -- (b3);
      \draw [arrow] (b3) -- (b4);
      \draw [arrow] (b4) -- node[anchor=south] {no} (b5);
      \draw [arrow] (b5) -- node[anchor=west] {no} (b6);
      \draw [arrow] (b5) -- node[anchor=south] {yes} (b7);
      \draw [arrow] (b4) --node[anchor=west] {yes} (15,2) -- (3,2) -- (b1);
      \draw [arrow] (b6) -| (b1);
      \draw [arrow] (b7) -- (23,6) -- (3,6) -- (b1);
      \draw [arrow] (b7) -- (stop);
      \draw [arrow] (b1) -- node[anchor=west] {yes} (kl);
      \draw [arrow] (kl) -| (stop);
  \end{tikzpicture}
  \end{figure}
  \end{frame}   
\end{document}