\documentclass{beamer}
\mode<presentation>
\setbeamertemplate{bibliography item}{}
\usepackage[utf8]{vietnam}
\usepackage{beamerthemesplit}
\usepackage{graphicx}
\usepackage{booktabs}
\usepackage{amsmath}
\usepackage{textpos}
\usepackage{pgfplots}
\usepackage{tikz}
\usepackage{hyperref}
\usepackage{caption}
\usetikzlibrary{shapes.geometric, arrows}
\usetikzlibrary {datavisualization} 
\pgfplotsset{compat=1.18, width = 7cm}
\usetikzlibrary{patterns}
\setbeamertemplate{bibliography item}[text]
\usetheme{Ilmenau} % AnnArbor, Ilmenau, Darmstadt, Dresden, CambridgeUS, Frankfurt, Singapore
\newtheorem{dn}{Định nghĩa}[section]
\newtheorem{dl}{Định lý}[section]
\newtheorem{tc}{Tính chất}[section]
\newtheorem{hq}{Hệ quả}[section]
\newtheorem{bd}{Bổ đề}[section]
\newtheorem{md}{Mệnh đề}[section]
\newtheorem{vd}{Ví dụ}[section]
\newtheorem{nx}{Nhận xét}[section]
\newcommand{\dom}{\text{{\rm dom}}}
\newcommand{\epi}{\text{{\rm epi}}}
\newcommand{\Min}{\text{{\rm Min}}}
\setbeamertemplate{theorems}[numbered]
\setbeamertemplate{definitions}[numbered]
\setbeamertemplate{footline}[frame number]
\usepackage{algorithm}
\usepackage{color}
\usepackage{algorithmic}
\usepackage{footmisc}
\usepackage{indentfirst} 
\usepackage{comment}
\AtBeginEnvironment{proof}{%
  \setbeamercolor{block title}{use=example text,fg=white,bg=example text.fg!75!black}
  \setbeamercolor{block body}{parent=normal text,use=block title example,bg=block title example.bg!10!bg}
}
\renewcommand{\thefootnote}{\arabic{footnote}}
\usefonttheme{professionalfonts}
\setbeamercolor{normal text}{bg=white,fg=black}
\renewcommand{\thefootnote}{\arabic{footnote}}
\beamertemplatetransparentcoveredhigh
\title[]{\fontsize{13pt}{10pt}\selectfont {\bf \LARGE Phương pháp nhánh cận}\\}
\author[]{\bf Nguyễn Chí Bằng \\}
\small{\date{\today}}

\begin{document}

\begin{frame}

\titlepage
\end{frame}

\begin{frame}{TÓM TẮT}
\begin{itemize}
\item Giới thiệu 2 mô hình chính trong phương pháp giải bài toán tối ưu tuyến tính nguyên:
\begin{itemize}
\item Tối ưu nguyên hoàn toàn
\item Tối ưu nguyên bộ phận.
\end{itemize}
\item Tập trung vào phương pháp giải bài toán tối ưu nguyên bộ phận thông qua thuật toán nhánh cận.
\begin{itemize}
\item Cơ sở lý thuyết.
\item Sơ đồ thuật toán.
\end{itemize}
\item Độ phức tạp của thuật toán.
\end{itemize}
\end{frame}

\begin{frame}
    \frametitle{NỘI DUNG}
    \tableofcontents
\end{frame}

\section{Giới thiệu}

\begin{frame}
   \center 
   \huge Giới thiệu bài toán 
\end{frame}

\begin{frame}{Tối ưu nguyên hoàn toàn (Pure integer linear program)}
    \begin{equation} \label{H}
        \begin{split}
        (H) \quad & z_h=c^Tx \quad \longrightarrow Max \\
                  & \left\{\begin{split}
                    &Ax \leq  b, \\
                    &x \geq 0, \text{ nguyên} \\
                    \end{split}\right.    
        \end{split}
        \end{equation}            
    \begin{itemize} \small
    \item Trong đó $c^T=(c_1 \: c_2 \: \ldots \: c_n)$, $A$ là ma trận $m\times n$, $b=\begin{pmatrix}
        b_1 \\
        b_2 \\
        \vdots \\
        b_m
        \end{pmatrix}$, với $x\in Z^n$.
    \item Bài toán $(H)$ gọi là bài toán \textbf{Tối ưu nguyên hoàn toàn.}
    \item Tập $S_h:=\{x\in Z^n_+: Ax\leq b\}$ là tập nghiệm của bài toán Tối ưu nguyên hoàn toàn.
    \end{itemize}
\end{frame}

%model_1: hoan toan
\begin{frame} {Minh hoạ bài toán}
    \begin{equation}
        \begin{split}
        \quad & 2x_1 + 2x_2 \quad \longrightarrow Max \\
                    & \left\{\begin{split}
                    & x_1 + 3x_2 \leq 24 \\
                    & \frac{13}{3}x_1 + 2x_2 \leq 32.5 \\
                    &x_1, x_2 \geq 0. \\
                    \end{split}\right.    
        \end{split}
        \end{equation}            
\end{frame}
%cho bài toán ví dụ rõ số liệu
\begin{frame}
\begin{center}
\begin{figure}
\begin{tikzpicture}
\begin{axis}
    [
    xmin=0,xmax=10,
    ymin=0,ymax=10,
    grid=both,
    grid style={line width=.1pt, draw=darkgray!50},
    major grid style={line width=.2pt,draw=darkgray!50},
    axis lines=middle,
    minor tick num=1,
    enlargelimits={abs=0},
    samples=100,
    domain = -20:20,
    ]
    \filldraw[green, pattern=north west lines, pattern color=green, line width=2pt] (0, 0) -- (0, 8) -- (4.5, 6.5) -- (7.5, 0) -- cycle;
\end{axis}
\end{tikzpicture}  
\caption{Tập nghiệm của bài toán Tối ưu nguyên hoàn toàn}
\end{figure}
\end{center}      
\end{frame}

\begin{frame}{Tối ưu nguyên bộ phận (Mixed integer linear program)}
\begin{equation}\label{B}
\begin{split}
(B) \quad & z_b=c^Tx+h^Ty \quad \longrightarrow Max \\
          & \left\{\begin{split}
            &Ax+Gy \leq  b, \\
            &x \geq 0, \text{ nguyên} \\
            &y \geq 0.
            \end{split}\right.    
\end{split}
\end{equation}    
\begin{itemize} \small
\item Trong đó $c^T=(c_1 \: c_2 \: \ldots \: c_n)$, $h^T=(h_1 \: h_2 \: \ldots \: h_p)$, $A$ là ma trận $m\times n$, $G$ là ma trận $m\times p$, $b=\begin{pmatrix}
    b_1 \\
    b_2 \\
    \vdots \\
    b_m
    \end{pmatrix}$, với $x\in Z^n$ và $y\in R^p$.
\item Bài toán $(B)$ gọi là bài toán \textbf{Tối ưu nguyên bộ phận.}
\item Tập $S_b:=\{(x,y)\in Z^n_+\times R^p_+: Ax+Gy\leq b\}$ là tập nghiệm của bài toán Tối ưu nguyên bộ phận.
\end{itemize}
\end{frame}
%cho bài toán ví dụ rõ số liệu
\begin{frame}{Minh hoạ bài toán}
   \begin{equation}
        \begin{split}
        \quad & x_1 + 2x_2 \quad \longrightarrow Max \\
                    & \left\{\begin{split}
                    & 5x_1 + \frac{15}{7}x_2 \leq 20 \\
                    & -2.4x_1 + \frac{30}{7}x_2 \leq 15 \\
                    &x_1, x_2 \geq 0. \\
                    \end{split}\right.    
        \end{split}
        \end{equation}            
\end{frame}

\begin{frame}
\begin{center}
\begin{figure}
\begin{tikzpicture}
\begin{axis}
    [
    xmin=0,xmax=5,
    ymin=0,ymax=5,
    grid style={line width=.1pt, draw=darkgray!50},
    major grid style={line width=.2pt,draw=darkgray!50},
    axis lines=middle,
    minor tick num=1,
    enlargelimits={abs=0},
    samples=100,
    domain = -20:20,
    xmajorgrids=true,
    ]
    \filldraw[green, pattern=north west lines, pattern color=green, line width=2pt] (0, 0) -- (0, 2.5) -- (2.5, 3.5) -- (4, 0) -- cycle;
\end{axis}
\end{tikzpicture}  
\caption{Tập nghiệm của bài toán Tối ưu nguyên bộ phận}
\end{figure}
\end{center}      
\end{frame}
    
\section{Mục tiêu}

\begin{frame}
   \center
   \huge Mục tiêu phương pháp
\end{frame}

\begin{frame}{Bài toán quan tâm}
\begin{equation}\label{P}
\begin{split}
(P) \quad & z_p=c^Tx+h^Ty \quad \longrightarrow Max \\
            & \left\{\begin{split}
                &Ax+Gy \leq  b, \\
                &x,y \geq 0.
            \end{split}\right.    
\end{split}
\end{equation}
\begin{itemize}
\item Trong đó $(P)$ là bài toán $(B)$ với nghiệm thuộc tập số thực.
\item Bài toán $(P)$ là một bài toán \textbf{Tối ưu tuyến tính thông thường} hay gọi đơn giản là bài toán Tối ưu tuyến tính (Natural linear programming relaxation).
\item Tập $S_p:=\{(x,y)\in R^n_+\times R^p_+: Ax+Gy\leq b\}$ là tập nghiệm của bài toán Tối ưu tuyến tính.
\end{itemize}
\end{frame}

\begin{frame}{Mục tiêu} %aka Lí do chọn bài toán
Giả sử ta nhận được tập phương án tối ưu của bài toán \eqref{B} sau hữu hạn lần giải, ký hiệu $(x_b, y_b)$ và giá trị tối ưu là $z_b$ thì ta có nhận xét sau:
\begin{nx} \label{nx}
\begin{itemize}
\item Nếu $S_b \subset S_p$ thì ta luôn nhận được $z_b \leq z_p$ và phương án có thể cải thiện.
\item Nếu $S_b = S_p$ thì ta nhận được $z_b = z_p$ và bài toán được giải.
\end{itemize}    
\end{nx}
Vì thế, ta chọn xử lý bài toán $(B)$ thông qua bài toán $(P)$ bằng cách cải thiện phương án thu được từ bài toán $(P)$ sao cho thoả điều kiện của bài toán $(B)$.
\end{frame}

\begin{frame}{Ví dụ}
    \begin{columns}
    \begin{column}{0.1\textwidth}
    \end{column}
    \begin{column}{0.4\textwidth}
        \begin{equation*}
        \begin{split}
            (P) \quad & 5.5x_1 + 2.1x_2 \quad \longrightarrow Max \\
            & \left\{\begin{split}
            & -1x_1 + x_2 \leq 2 \\
            & 8x_1 + 2x_2 \leq 17 \\
            &x_1 \geq 0, \\
            &x_2 \geq 0. \\
            \end{split}\right. \\
        \end{split}
        \end{equation*}
    \end{column}
    \begin{column}{0.4\textwidth}
        \begin{equation*}
            \Longrightarrow
            \left\{\begin{split}
            \color{red} x_1 &\color{red} = 1.3 \\
            \color{red} x_2 &\color{red} = 3.3 \\
            z&=14.08
        \end{split}\right.
        \end{equation*}
    \end{column}
    \begin{column}{0.1\textwidth}
    \end{column}
    \end{columns}
    \vspace{1cm}
    \center
    \Large
    Phương án có thể cải thiện.
\end{frame}


\begin{frame}{Ví dụ}
    \begin{columns}
    \begin{column}{0.1\textwidth}
    \end{column}
    \begin{column}{0.4\textwidth}
        \begin{equation*}
        \begin{split}
            (P) \quad & 3x_1 + 4x_2 \quad \longrightarrow Max \\
            & \left\{\begin{split}
            & 2.5x_1 + \frac{15}{4}x_2 \leq 20 \\
            & x_1 + \frac{5}{3}x_2 \leq \frac{50}{3} \\
            &x_1 \geq 0, \\
            &x_2 \geq 0. \\
            \end{split}\right. \\
        \end{split}
        \end{equation*}
    \end{column}
    \begin{column}{0.4\textwidth}
        \begin{equation*}
            \Longrightarrow
            \left\{\begin{split}
            \color{blue} x_1 &\color{blue} = 5 \\
            \color{blue} x_2 &\color{blue} = 7 \\
            z& =43
        \end{split}\right.
        \end{equation*}
    \end{column}
    \begin{column}{0.1\textwidth}
    \end{column}
    \end{columns}
    \vspace{1cm}
    \center
    \Large
    Bài toán được giải.
\end{frame}

\section{Thuật toán nhánh cận}
\begin{frame}
   \center
   \huge Thuật toán nhánh cận \\ (Branch-and-Bound)
\end{frame}

\begin{frame}{Nội dung chính}
\large
Ở mục này, chúng ta sẽ làm rõ những ý tưởng sau:
\begin{itemize}
\item Định nghĩa cận và xác định cận của bài toán.
\item Phương pháp xử lý bài toán nhỏ kèm ví dụ minh hoạ.
\item Sơ đồ thuật giải và các ví dụ minh hoạ cho bài toán lớn.
\end{itemize}
\end{frame}

\begin{frame}{Phương pháp xác định cận}
Ta gọi $x_j$ với $1 \leq j \leq n$ là nghiệm thu được từ bài toán $(P)$.
\begin{dl}\label{cmnguyen}
\begin{itemize}
\item Với mỗi $x_j \in \mathbb{R}$, tồn tại duy nhất số nguyên $k \in \mathbb{Z}$ sao cho $k \leq x_j < k+1$.
\begin{itemize}
\item Giá trị $k$ khi đó ta gọi là phần nguyên nhỏ nhất của $x_j$, ký hiệu là $\lfloor x_j \rfloor$.
\item Giá trị $k+1$ gọi là phần nguyên lớn nhất của $x_j$, ký hiệu là $\lceil x_j \rceil$.
\end{itemize}
\end{itemize}
\end{dl}
\end{frame}

\begin{frame}
\begin{proof}
\begin{itemize}
\item $\forall x_j \geq 0$, ta có:
\begin{itemize}
\item $x_j \in \mathbb{Z} \Rightarrow \lfloor x_j \rfloor =x_j$. (đpcm)
\item $x_j \notin \mathbb{Z}$, ta đặt $S=\{ m \in \mathbb{N} \: \vert \: m > x_j \}$ \\ Theo tiên đề Peano, luôn tồn tại một min$S$, vì thế ta dễ dàng nhận thấy min$S -1=k=\lfloor x_j \rfloor$ với $k \in \mathbb{Z}$ và thoả $k \leq x_j < k+1$ hay $\lfloor x_j \rfloor \leq x_j <\lceil x_j \rceil$. 
\end{itemize}
\end{itemize}
\end{proof}

\begin{vd}
Ta có $x_1=3.3$, vậy khi đó phần nguyên nhỏ nhất của $x_1$ là $\lfloor x_1 \rfloor = 3$ và phần nguyên lớn nhất là $\lceil x_1 \rceil =4$.
\end{vd}
\end{frame}

\begin{frame}{Phương pháp xử lý bài toán}
\large
\begin{itemize}
\item Từ \eqref{nx} và \eqref{cmnguyen}, ta thấy rằng nếu $\exists x_j \notin \mathbb{Z}$, thì ta có thể tiếp tục cải thiện phương án cho đến khi $\forall x_j \in \mathbb{Z}$. 
\item Nếu nghiệm thu được là $x_j \notin \mathbb{Z}$ ta thiết lập được 2 bài toán con từ bài toán $(P)$ ban đầu, ký hiệu $(P_1)$ và $(P_2)$.
\end{itemize}
\end{frame}

\begin{frame}
\begin{equation}
    \begin{split}
    (P_1) \quad & z_1=c^Tx+h^Ty \quad \longrightarrow Max \\
                & \left\{\begin{split}
                    &Ax+Gy \leq  b \\
                    &\color{blue} x_j \leq \lfloor x_j \rfloor , \\
                    &x,y \geq 0.
                \end{split}\right.    
    \end{split}
\end{equation}
\begin{itemize}
\item Tập $S_1:=S_p \cap \{ (x,y): x_j \leq \lfloor x_j \rfloor \}$ là tập nghiệm tối ưu của bài toán con $(P_1)$.
\end{itemize}
\end{frame}

\begin{frame}
\begin{equation}
    \begin{split}
    (P_2) \quad & z_2=c^Tx+h^Ty \quad \longrightarrow Max \\
                & \left\{\begin{split}
                    &Ax+Gy \leq  b \\
                    &\color{blue} x_j \geq \lceil x_j \rceil , \\
                    &x,y \geq 0.
                \end{split}\right.    
    \end{split}
\end{equation}
\begin{itemize}
\item Tập $S_2:=S_p \cap \{ (x,y): x_j \geq \lceil x_j \rceil \}$ là tập nghiệm tối ưu của bài toán con $(P_2)$.
\end{itemize}
\end{frame}

\begin{frame}{Điều kiện nghiệm}
\begin{itemize}
\item Nếu tồn tại $(P_i)$ với $i=1,2$ không giải được $(S_i = \emptyset )$, ta gọi bài toán \textbf{vô nghiệm}.
\item Giả sử $x^i$ là nghiệm tối ưu của bài toán $(P_i)$ và giá trị tối ưu là $z_i$ với $i = 1,2$.
\begin{itemize}
\item Nếu $\forall x^i \in Z^n_+$, ta nói $S_i$ là tập nghiệm thoả mãn bài toán tối ưu nguyên bộ phận, $z^*_i$ là giá trị tối ưu và bài toán con $(P_i)$ được giải \textbf{(gọt bởi nghiệm nguyên)}.
\item Nếu $\exists x^i \notin Z^n_+$ đồng thời $z_i \leq z^*_i$, ta dừng phân nhánh và bỏ qua bài toán \textbf{(gọt bởi cận)}.
\item Nếu $\exists x^i \notin Z^n_+$ đồng thời $z_i > z^*_i$, bài toán chưa tối ưu và có thể tiếp tục cải thiện.
\iffalse
Với $x^i_{j^'}$ là nghiệm không nguyên và phân thành 2 bài toán con $P_3$ với tập nghiệm $S_l := S_i \cap \{ (x,y): x_{j^{'}} \leq \lfloor x_{j^{'}} \rfloor \}$, $P_l$ với $S_l := S_i \cap \{ (x,y): x_{j^{'}} \geq \lceil x_{j^{'}} \rceil \}$ trong đó $l=3,4$ và lặp lại quá trình.
\fi
\end{itemize}
\end{itemize}
\end{frame}

\begin{frame}{Ví dụ minh hoạ}
    \begin{equation*}
        \begin{split}
            (P) \quad z_p= & 5.5x_1 + 2.1x_2 \quad \longrightarrow Max \\
            & \left\{\begin{split}
            & -x_1 + x_2 \leq 2 \\
            & 8x_1 + 2x_2 \leq 17 \\
            &x_1 \geq 0, \\
            &x_2 \geq 0. \\
            \end{split}\right. \\
        \end{split}
    \end{equation*}
\end{frame}

\begin{frame}
    Giải bài toán bằng phương pháp đơn hình thông thường ta được nghiệm $x_1 =1.3$, $x_2 = 3.3$ và $z_p=14.08$.
    \vspace{1cm}
    \begin{figure}
    \begin{tikzpicture}[scale=.9]
    \begin{axis}
        [
        xmin=0,xmax=3,
        ymin=0,ymax=5,
        grid style={line width=.1pt, draw=darkgray!50},
        major grid style={line width=.2pt,draw=darkgray!50},
        axis lines=middle,
        minor tick num=1,
        enlargelimits={abs=0},
        samples=100,
        domain = -20:20,
        xmajorgrids=true,
        ]
        \filldraw[blue, pattern=north west lines, pattern color=blue, line width=2pt] (0, 0) -- (0, 2) -- (1.3, 3.3) -- (2.125, 0) -- cycle;
    \end{axis}
    \end{tikzpicture}  
    \caption{Tập nghiệm của bài toán}
    \end{figure}
\end{frame}

%template to copy-paste
\begin{frame}
Chọn $x_1=1.3$ để cải thiện phương án, ta thu được 2 bài toán con sau:
\begin{columns}
\begin{column}{0.5\textwidth}
    \begin{equation*}
        \begin{split}
            (P_1) \quad z_1= & 5.5x^1_1 + 2.1x^1_2 \\
            & \left\{\begin{split}
            & -x^1_1 + x^1_2 \leq 2 \\
            & 8x^1_1 + 2x^1_2 \leq 17 \\
            & \color{blue} x^1_1 \leq 1 \\
            &x^1_1 \geq 0 \\
            &x^1_2 \geq 0. \\
            \end{split}\right. \\
        \end{split}
    \end{equation*}
\end{column}
\begin{column}{0.5\textwidth}
   \begin{equation*}
        \begin{split}
            (P_2) \quad z_2= & 5.5x^2_1 + 2.1x^2_2  \\
            & \left\{\begin{split}
            & -x^2_1 + x^2_2 \leq 2 \\
            & 8x^2_1 + 2x^2_2 \leq 17 \\
            & \color{blue} x^2_1 \geq 2 \\
            &x^2_1 \geq 0 \\
            &x^2_2 \geq 0. \\
            \end{split}\right. \\
        \end{split}
    \end{equation*}
\end{column}
\end{columns}
\end{frame}

\begin{frame}
        \begin{equation*}
        \begin{split}
            (P_1) \quad z_1= & 5.5x^1_1 + 2.1x^1_2 \quad \longrightarrow Max \\
            & \left\{\begin{split}
            & -x^1_1 + x^1_2 \leq 2 \\
            & 8x^1_1 + 2x^1_2 \leq 17 \\
            & \color{blue} x^1_1 \leq 1 \\
            &x^1_1 \geq 0 \\
            &x^1_2 \geq 0. \\
            \end{split}\right. \\
        \end{split}
    \end{equation*}
\end{frame}

\begin{frame}
    Giải bài toán ta được $x^1_1=1, x^1_2=3$ và $z_1=11.8$. Bài toán được giải \textbf{(gọt bởi nghiệm nguyên)}.
    \vspace{1cm}
    \begin{figure}
    \begin{tikzpicture}[scale=.9]
    \begin{axis}
        [
        xmin=0,xmax=3,
        ymin=0,ymax=5,
        grid style={line width=.1pt, draw=darkgray!50},
        major grid style={line width=.2pt,draw=darkgray!50},
        axis lines=middle,
        minor tick num=1,
        enlargelimits={abs=0},
        samples=100,
        domain = -20:20,
        xmajorgrids=true,
        ]
        \filldraw[blue, pattern=north west lines, pattern color=blue, line width=2pt] (0, 0) -- (0, 2) -- (1, 3) -- (1, 0) -- cycle;
    \end{axis}
    \end{tikzpicture}  
    \caption{Tập nghiệm của bài toán $(P_1)$}
    \end{figure}
\end{frame}

\begin{frame}
    Tương tự bài toán $(P_2)$ ta được $x^2_1 = 2, x^2_2 = 0.5$ và $z_2=12.05$. Ta chọn $x^2_2 = 0.5$ để cải thiện phương án. Ta được 2 bài toán con $(P_3)$ và $(P_4)$:
    \begin{columns}
\begin{column}{0.5\textwidth}
    \begin{equation*}
        \begin{split}
            (P_3) \quad z_3= & 5.5x^3_1 + 2.1x^3_2 \\
            & \left\{\begin{split}
            & -x^3_1 + x^3_2 \leq 2 \\
            & 8x^3_1 + 2x^3_2 \leq 17 \\
            & \color{blue} x^3_1 \geq 2 \\
            & \color{blue} x^3_2 \leq 0 \\
            &x^3_1 \geq 0 \\
            &x^3_2 \geq 0. \\
            \end{split}\right. \\
        \end{split}
    \end{equation*}
\end{column}
\begin{column}{0.5\textwidth}
   \begin{equation*}
        \begin{split}
            (P_4) \quad z_4= & 5.5x^4_1 + 2.1x^4_2  \\
            & \left\{\begin{split}
            & -x^4_1 + x^4_2 \leq 2 \\
            & 8x^4_1 + 2x^4_2 \leq 17 \\
            & \color{blue} x^4_1 \geq 2 \\
            & \color{blue} x^4_2 \geq 1 \\
            &x^4_1 \geq 0 \\
            &x^4_2 \geq 0. \\
            \end{split}\right. \\
        \end{split}
    \end{equation*}
\end{column}
\end{columns}
\end{frame}

\begin{frame}
    \begin{itemize} 
    \item Giải bài toán $(P_3)$ ta được $x^3_1=2.125, x^3_2=0$ và $z_3=11.6875 \Rightarrow $ không khả thi do $z_3 < z_1$ \textbf{(gọt bởi cận)}. 
    \item Bài toán $(P_4)$ \textbf{vô nghiệm}.
    \item Vậy phương án tối ưu của bài toán là $x^1_1=1, x^1_2=3$ và $z=11.8$.
\end{itemize}
\end{frame}

\begin{frame}{Sơ đồ thuật toán}
\begin{itemize}
\item Ta gọi bài toán $(P)$ có nút ban đầu là $N_0$, tương ứng mỗi bài toán tối ưu tuyến tính thông thường $(P_i)$ ứng với mỗi nút $N_i$ trên sơ đồ nhánh và $\mathcal{L}$ là danh sách chứa các nút được lập thông qua lý thuyết xác định cận và lý thuyết nghiệm.
\item Ta đánh dấu giá trị tối ưu tốt nhất và nghiệm tối ưu tốt nhất của bài toán lần lượt là $z^*$ và $(x^*,y^*)$.
\end{itemize}
\end{frame}

\begin{frame}{Sơ đồ thuật toán}
\small
\setlength{\parindent}{4em}
\noindent \textbf{Bước 1. Thiết lập} \\
Đặt $\mathcal{L}:=\{N_0 \}$, $z^*=z_p$ và $(x^*,y^*)=(x,y)$. \\
\noindent \textbf{Bước 2. Kiểm tra} \\
Nếu $\mathcal{L} = \emptyset$ thì nghiệm tối ưu của bài toán là $(x^*,y^*)$, giá trị \\ tối ưu là $z^*$ và bài toán được giải. \\
Nếu $\mathcal{L} \neq \emptyset$, chuyển sang bước 3. \\ 
\noindent \textbf{Bước 3. Chọn nút} \\
Chọn nút $N_i$ từ danh sách $\mathcal{L}$ và xoá khỏi $\mathcal{L}$ sau đó chuyển sang \\ bước 4. \\ 
\noindent \textbf{Bước 4. Xác định cận} \\ 
Giải bài toán $(P_i)$, nếu bài toán vô nghiệm hoặc $z_i \leq z^*$, quay \\ lại bước 2, nếu không, chuyển sang bước 5. \\ 
\noindent \textbf{Bước 5. Gọt nghiệm} \\
Nếu tồn tại $x^i \notin Z^n_+$, ta thêm nút $N_{i+1}, \ldots , N_{k}$ vào $\mathcal{L}$ và quay \\ về bước 2. \\
Nếu không tồn tại $x^i \notin Z^n_+$, tức $\forall x^i \in Z^n_+$, ta đặt $z_i = z^*$, \\ $(x^i,y^i) = (x^*,y^*)$ và quay lại bước 2.
\end{frame}    

\begin{frame}{Sơ đồ thuật toán}
\center
\begin{figure}
\begin{tikzpicture}[scale=0.33, every node/.style={scale=0.33}, node distance=4cm]
% Define styles
\tikzstyle{startstop} = [rectangle, rounded corners, minimum width=1cm, minimum height=1cm, align=center, draw=black, fill=red!30]
\tikzstyle{process} = [rectangle, minimum width=1cm, minimum height=1cm, align=center, draw=black, fill=orange!30]
\tikzstyle{decision} = [diamond, aspect=2, minimum width=1cm, minimum height=1cm, align=center, draw=black, fill=green!30]
\tikzstyle{io} = [trapezium, trapezium left angle=70, trapezium right angle=110, minimum width=1cm, minimum height=1cm, align=center, draw=black, fill=blue!30]
\tikzstyle{arrow} = [thick,->,>=stealth,line width=0.2pt]
% Place nodes
\node (start) [startstop, xshift=-5cm] {Bắt đầu};
\node (b0) [io, right of=start] {$\mathcal{L}:=\{N_0 \}$ \\ $z^*=z_p$ \\ $(x^*,y^*)=(x,y)$};
\node (b1) [decision, right of=b0] {$\mathcal{L} = \emptyset$?};
\node (b2) [process, right of=b1] {Chọn $N_i$ từ $\mathcal{L}$ \\ và xoá khỏi $\mathcal{L}$};
\node (b3) [process, right of=b2] {Giải $(P_i)$};
\node (b4) [decision, right of=b3] {Vô nghiệm?};
\node (b5) [decision, right of=b4] {$\exists x^i \notin Z^n_+$?};
\node (b6) [io, above of=b5] {$z_i = z^*$, $(x^i,y^i) = (x,y)$};
\node (b7) [io, right of=b5] {Thêm \\ $N_{i1}, \ldots , N_{ik}$ \\ vào $\mathcal{L}$};
\node (kl) [process, below of=b1] {bài toán \\ được giải};
\node (stop) [startstop, right of=b7] {Kết thúc};
% Draw arrows
\draw [arrow] (start) -- (b0);
\draw [arrow] (b0) -- (b1);
\draw [arrow] (b1) -- node[anchor=south] {no} (b2);
\draw [arrow] (b2) -- (b3);
\draw [arrow] (b3) -- (b4);
\draw [arrow] (b4) -- node[anchor=south] {no} (b5);
\draw [arrow] (b5) -- node[anchor=west] {no} (b6);
\draw [arrow] (b5) -- node[anchor=south] {yes} (b7);
\draw [arrow] (b4) --node[anchor=west] {yes} (15,2) -- (3,2) -- (b1);
\draw [arrow] (b6) -| (b1);
\draw [arrow] (b7) -- (23,6) -- (3,6) -- (b1);
\draw [arrow] (b7) -- (stop);
\draw [arrow] (b1) -- node[anchor=west] {yes} (kl);
\draw [arrow] (kl) -| (stop);
\end{tikzpicture}
\caption{Lưu đồ giải thuật của thuật toán nhánh cận.}
\end{figure}
\end{frame}   

\begin{frame}{Ví dụ minh hoạ}
    \begin{equation*}
    \begin{split}
    (P) \quad z_p= & 5.5x_1 + 2.1x_2 + 3x_3 \quad \longrightarrow Max \\
    & \left\{\begin{split}
    & -x_1 + x_2 + x_3 \leq 2 \\
    & 8x_1 + 2x_2 + x_3\leq 17 \\
    & 9x_1 + x_2 + 6x_3 \leq 20 \\
    &x_i \geq 0 , \: i=\overline{\rm 1, \ldots ,3} \\
    \end{split}\right. \\
    \end{split}
    \end{equation*}
    Giải bài toán $(P)$ ta được $x_1=1.38, x_2=2.55, x_3=0.83$. \\
    Ta chọn biến $x_1$ để phân nhánh. ($x_2, x_3$ tương tự)
    \end{frame}
    
    \begin{frame}
    Với $x^1_1 \leq 1$ \\ 
    \begin{equation*}
    \begin{split}
      (P_1) \quad z_p= & 5.5x^1_1 + 2.1x^1_2 + 3x^1_3 \quad \longrightarrow Max \\
      & \left\{\begin{split}
      & -x^1_1 + x^1_2 + x^1_3 \leq 2 \\
      & 8x^1_1 + 2x^1_2 + x^1_3\leq 17 \\
      & 9x^1_1 + x^1_2 + 6x^1_3 \leq 20 \\
      & x^1_1 \leq 1 \\
      &x^1_i \geq 0 , \: i=\overline{\rm 1, \ldots ,3} \\
      \end{split}\right. \\
    \end{split}
    \end{equation*}
    $\rightarrow z_1=13.24, x^1_1=1, x^1_2=1.4, x^1_3=1.6$
    \end{frame}
    
    \begin{frame}
    Với $x^2_1 \geq 2$ \\
    \begin{equation*}
      \begin{split}
        (P_2) \quad z_p= & 5.5x^2_1 + 2.1x^2_2 + 3x^2_3 \quad \longrightarrow Max \\
        & \left\{\begin{split}
        & -x^2_1 + x^2_2 + x^2_3 \leq 2 \\
        & 8x^2_1 + 2x^2_2 + x^2_3\leq 17 \\
        & 9x^2_1 + x^2_2 + 6x^2_3 \leq 20 \\
        & x^2_1 \geq 2 \\
        &x^2_i \geq 0 , \: i=\overline{\rm 1, \ldots ,3} \\
        \end{split}\right. \\
    \end{split}
    \end{equation*}
    $\rightarrow z_2=12.58, x^2_1=2, x^2_2=0.36, x^2_3=0.27$
    \end{frame}
    
    \begin{frame}
      Ta tiếp tục chọn $x^1_2=1.4$ từ $(P_1)$. Với $x^3_2 \leq 1$
    \begin{equation*}
      \begin{split}
          (P_3) \quad z_p= & 5.5x^3_1 + 2.1x^3_2 + 3x^3_3 \quad \longrightarrow Max \\
          & \left\{\begin{split}
          & -x^3_1 + x^3_2 + x^3_3 \leq 2 \\
          & 8x^3_1 + 2x^3_2 + x^3_3\leq 17 \\
          & 9x^3_1 + x^3_2 + 6x^3_3 \leq 20 \\
          & x^3_1 \leq 1 \\
          & x^3_2 \leq 1 \\
          &x^3_i \geq 0 , \: i=\overline{\rm 1, \ldots ,3} \\
          \end{split}\right. \\
      \end{split}
    \end{equation*}
    $\rightarrow z_3=12.6, x^3_1=1, x^3_2=1, x^3_3=1.66$
    \end{frame}
    
    \begin{frame}
     Với $x^4_2 \geq 2$
    \begin{equation*}
      \begin{split}
          (P_4) \quad z_p= & 5.5x^4_1 + 2.1x^4_2 + 3x^4_3 \quad \longrightarrow Max \\
          & \left\{\begin{split}
          & -x^4_1 + x^4_2 + x^4_3 \leq 2 \\
          & 8x^4_1 + 2x^4_2 + x^4_3\leq 17 \\
          & 9x^4_1 + x^4_2 + 6x^4_3 \leq 20 \\
          & x^4_1 \leq 1 \\
          & x^4_2 \geq 2 \\
          &x^4_i \geq 0 , \: i=\overline{\rm 1, \ldots ,3} \\
          \end{split}\right. \\
      \end{split}
    \end{equation*}
    $\rightarrow z_4=12.7, x^4_1=1, x^4_2=2, x^4_3=1 \rightarrow$ \textbf{gọt bởi nghiệm nguyên}.
    \end{frame}
    
    \begin{frame}
    Ta chọn $x^3_3=1.66$ từ $(P_3)$. Với $x^5_3 \leq 1$.
    \begin{equation*}
      \begin{split}
          (P_5) \quad z_p= & 5.5x^5_1 + 2.1x^5_2 + 3x^5_3 \quad \longrightarrow Max \\
          & \left\{\begin{split}
          & -x^5_1 + x^5_2 + x^5_3 \leq 2 \\
          & 8x^5_1 + 2x^5_2 + x^5_3\leq 17 \\
          & 9x^5_1 + x^5_2 + 6x^5_3 \leq 20 \\
          & x^5_1 \leq 1 \\
          & x^5_2 \leq 1 \\
          & x^5_3 \leq 1 \\
          &x^5_i \geq 0 , \: i=\overline{\rm 1, \ldots ,3} \\
          \end{split}\right. \\
      \end{split}
    \end{equation*}
    $\rightarrow z_5=10.6, x^5_1=1, x^5_2=1, x^5_3=1 \rightarrow$ gọt bởi nghiệm nguyên nhưng $z_5<z_4 \rightarrow$ loại.
    \end{frame}
    
    \begin{frame}
    Với $x^6_3 \geq 2$.
    \begin{equation*}
      \begin{split}
          (P_6) \quad z_p= & 5.5x^6_1 + 2.1x^6_2 + 3x^6_3 \quad \longrightarrow Max \\
          & \left\{\begin{split}
          & -x^6_1 + x^6_2 + x^6_3 \leq 2 \\
          & 8x^6_1 + 2x^6_2 + x^6_3\leq 17 \\
          & 9x^6_1 + x^6_2 + 6x^6_3 \leq 20 \\
          & x^6_1 \leq 1 \\
          & x^6_2 \leq 1 \\
          & x^6_3 \geq 2 \\
          &x^6_i \geq 0 , \: i=\overline{\rm 1, \ldots ,3} \\
          \end{split}\right. \\
      \end{split}
    \end{equation*}
    $\rightarrow z_6=12.08, x^6_1=0.8, x^6_2=0.8, x^6_3=2 \rightarrow$ \textbf{gọt bởi cận}.
    \end{frame}
    
    \begin{frame}
      Ta chọn $x^2_2=0.36$ từ $(P_2)$
      \begin{itemize}
        \item $(P_7)$ cho $z_7=12.1, x^7_1=2.1, x^7_2=0, x^7_3=0.17$, do $z_7<z_4 \rightarrow$ loại.
        \item $(P_8)$ cho kết quả \textbf{vô nghiệm}.
      \end{itemize} 
    Vậy nghiệm tối ưu của bài toán là $x_1=1, x_2=2, x_3=1$ và giá trị tối ưu $z=12.7$.
    \end{frame}    

\section{Độ phức tạp}

\begin{frame}
\end{frame}

\end{document}