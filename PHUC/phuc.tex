\documentclass{beamer}
\usepackage[utf8]{vietnam}
\usetheme{Madrid}
\usecolortheme{default}
\usepackage{tabularx}
\title
{BÁO CÁO NGHIÊN CỨU KHOA HỌC}
\subtitle{
MỘT SỐ BẤT ĐẲNG THỨC ĐỐI VỚI ĐA THỨC PHỨC VÀ ỨNG DỤNG }
\institute[VFU]
{Giảng viên hướng dẫn: \vspace{0.3cm} \textbf{PGS.TS KIỀU PHƯƠNG CHI}\par
Chủ nhiệm đề tài: \textbf{MAI HOÀNG PHÚC}}
\begin{document}
\frame{\titlepage}
\begin{frame}
\frametitle{I.ĐẶT VẤN ĐỀ CẦN GIẢI QUYẾT}
\begin{block}{Bài toán:}
 Cho một đa thức phức bậc n có các nghiệm nằm trong một hình tròn ∣z∣=R. Tìm giá trị cực đại của đạo hàm của đa thức đó trên hình tròn.
\end{block}
\vspace{0.5cm}
Giải pháp thông thường cho bài toán này là sử dụng định lý Lagrange để tìm các điểm cực đại và cực tiểu của đạo hàm của đa thức đó. Tuy nhiên, giải pháp này có thể rất phức tạp, đặc biệt là khi đa thức có bậc cao.\par
Bằng cách sử dụng bất đẳng thức mới, chúng ta có thể giải bài toán này một cách đơn giản hơn'.
\end{frame}
\begin{frame}
Theo bất đẳng thức mới, ta có:
\[max_{|z|=R} |P(z)| \ge \frac{1}{2}max_{|z|=R} 
|P'(z)|\]
Để tính giá trị cực đại của đạo hàm của đa thức đó trên hình tròn, ta cần tính giá trị cực đại trên hình tròn. Điều này có thể thực hiện bằng cách sử dụng định lý Jensen.\par
Theo định lý Jensen ta có:
\[max_{|z|=R}|P(z)| \ge \frac{1}{n}\sum_{k=1}^n |a_k|R^k\]
Do đó, ta có:
\[max_{|z|=R}|P(z)| \ge \frac{1}{2}.\frac{1}{n}\sum_{k=1}^n |a_k|R^k\]
Từ đó, ta có:
\[max_{|z|=R} |P(z)| \ge \frac{|a_1R|}{2n}\]
\end{frame}
\begin{frame}{Kết luận}
\begin{block}{Ta được:}
 Giải pháp này cho ta một giới hạn dưới cho giá trị cực đại của đạo hàm của đa thức đó trên hình tròn. Giới hạn này có thể được sử dụng để đánh giá độ lớn của đạo hàm của đa thức đó. Đó cũng là điểm mạnh của bất đẳng thức trên khi đánh giá về nghiệm của đạo hàm của đa thức phức với độ phức tạp cao. 
\end{block}


\end{frame}


\begin{frame}
\frametitle{II.TÍNH CẤP THIẾT CỦA ĐỀ TÀI}
{\begin{itemize}
\begin{block}{}
Đa thức phức có vai trò rất lớn đối với nhiều chuyên ngành của toán học, nó vừa là công cụ trong nhiều lĩnh vực như đại số, giải tích, hình học và là đối tượng nghiêm cứu trong một số lĩnh vực chuyên sâu của toán học.\par
\vspace{0.3cm}
\par Ví dụ:
\begin{itemize}
\item{Lý thuyết số:}
Bất đẳng thức Hermite-Hadamard có thể được sử dụng để chứng minh định lý Dirichlet về sự phân bố của các số nguyên nguyên tố.
\end{itemize}
\begin{itemize}
\item{Giải tích phức:}
Bất đẳng thức Cauchy có thể được sử dụng để chứng minh định lý Rounché.
\end{itemize}
\begin{itemize}
\item{Lý thuyết điều khiển:}
Bất đẳng thức Bezout có thể được sử dụng để nghiêm cứu các tính chất của các hệ thống điều khiển.
\end{itemize}
\end{block}
\end{itemize}}
\end{frame}


\begin{frame}{}
\begin{block}{}
-Việc khảo sát tính chất của đa thức phức là khá đa dạng,  liên quan đến nhiều lĩnh vực và chuyên ngành.\par
\vspace{0.2cm}
-Một trong những hướng nghiên cứu có nhiều ứng dụng trong giải tích phức và một số chủ đề liên quan đó là khảo sát các bất đẳng thức đối với đa thức phức trên các tập bị chặn của mặt phẳng phức, hoặc các đánh giá đối với tập nghiệm của các đa thức.\par
\vspace{0.2cm}
-Nhằm tập dượt nghiên cứu khoa học và tìm hiểu sâu hơn về một số bất đẳng thức đối với đa thức phức và ứng dụng, chúng tôi lựa chọn đề tài nghiên cứu: Về một số bất đẳng thức đối với đa thức phức và ứng dụng.
\end{block}  
\end{frame}




\begin{frame}
\frametitle{III.TÌNH HÌNH NGHIÊN CỨU LIÊN QUAN ĐẾN ĐỀ TÀI}
\begin{block}{}
-Các bất đẳng thức đối với đa thức phức được thiết lập phong phú trong nhiều hoàn cảnh, chẳng hạn các đánh giá về nghiệm của đa thức phức thông qua các hệ số của nó, hoặc ngược lại.\vspace{0.3cm}\par -Hay đánh giá của đối với đạo hàm của đa thức trên các đĩa đóng tâm tại gốc của mặt phẳng phức. Nó là sự mở rộng của nhiều kết quả cổ điển được thiết lập bởi Bernstein, Erdos, Lax,…  
\end{block}
\end{frame}
\begin{frame}{ NHÀ TOÁN HỌC}
\begin{center}
    \begin{figure}[htp]
    \begin{center}
     \includegraphics[scale=0]{berstein.jpg}
\begin{figure}
        \centering
        \includegraphics[width=0.65\linewidth]{berstein.jpg}
        \caption{\bf{Sergei Bernstein}}
        \label{fig:enter-label}
    \end{figure}
    \end{center}
    \end{figure}
\end{center}
\end{frame}
\begin{frame}{ NHÀ TOÁN HỌC}
\begin{center}
    \begin{figure}[htp]
    \begin{center}
     \includegraphics[scale=0]{berstein.jpg}
\begin{figure}
        \centering
        \includegraphics[width=0.65\linewidth]{lax.jpg}
        \caption{\bf{Peter David Lax}}
        \label{fig:enter-label}
    \end{figure}
    \end{center}
    \end{figure}
\end{center}
\end{frame}

\begin{frame}
\frametitle{IV.ĐỐI TƯỢNG VÀ PHẠM VI NGHIÊN CỨU}
\begin{block}{}
    Đối tượng nghiên cứu là các đa thức.\par Phạm vi nghiên cứu là một số bất đẳng thức đối với đa thức phức và ứng dụng.
\end{block}
\end{frame}
\begin{frame}
\frametitle{V.MỤC TIÊU NGHIÊN CỨU VÀ PHƯƠNG PHÁP NGHIÊM CỨU}
\begin{block}{Mục tiêu của đề tài:}
Tìm hiểu tài liệu để khảo sát một cách có hệ thống một số bất đẳng thức đối với đa thức phức và đạo hàm của chúng. Từ đó tìm cách mở rộng thêm các bất đẳng thức đó và tìm các ứng dụng trong việc khảo sát một số tính chất quen thuộc của đa thức: tính bất khả quy, khả quy, ước lượng nghiệm,…
\end{block}
\begin{block}{Phương pháp:}
    Sử dụng các phương pháp nghiên cứu lý thuyết, thông qua tìm hiểu tài liệu để khảo sát một cách có hệ thống một số bất đẳng thức đối với đa thức phức.
\end{block}
\end{frame}
\begin{frame}{Reference}
\begin{itemize}
    \item {Aziz, A., Rather, N.A. (1998). A refinement of a theorem of Paul Turan concerning polynomials. Math. Inequal. Appl., 1, 231–238.}
    \item{Frappier, C., Rahman, Q., Ruscheweyh, S.T. (1985). New inequalities for polynomials. Trans. Am. Math. Soc., 288, 60-99.}
    \item{Govil, N.K. (1973). On the derivative of a polynomial. Proc. Am. Math. Soc., 41, 543-546.}
    \item{Govil, N.K. Inequalities for the derivative of a polynomial. 3. Approx. Theory, 63, 65-71 (1990).}
    \item{ Lax, P.D. (1944). Proof of a conjecture due to Eason the derivative of a polynomial. Bull. Am. Math. Soc., 50, 509-513.}
    \item{Milovic, G.V., Mitrinovic, D.S., Rassias, T.M. (1994). Topics in Poly- nomials: Extremal Properties, Inequalities, Zeros. World Scientific Publishing co., Singapore.}
\end{itemize}
    
\end{frame}
\end{document}