\documentclass{article}
\usepackage[utf8]{inputenc}
\usepackage[vietnamese]{babel}
\usepackage{amsmath,amssymb,amsfonts}
\usepackage{graphicx}
\usepackage{subcaption}
\usepackage{float}
\usepackage{hyperref}
\usepackage{blindtext}
\usepackage{titlesec}
\usepackage{array}
\usepackage{mathtools}
\usepackage{tabularx}
\usepackage{geometry}
\usepackage{nccmath}
\newtheorem{theorem}{Theorem}[section]
\newtheorem{corollary}{Corollary}[theorem]
\newtheorem{lemma}[theorem]{Lemma}
\setlength\tabcolsep{10pt}
\setlength\arraycolsep{10pt}
\setlength\extrarowheight{10pt}
\renewcommand{\labelenumii}{\arabic{enumi}.\arabic{enumii}}
\renewcommand{\labelenumiii}{\arabic{enumi}.\arabic{enumii}.\arabic{enumiii}}
\renewcommand{\labelenumiv}{\arabic{enumi}.\arabic{enumii}.\arabic{enumiii}.\arabic{enumiv}}
\newtheorem{dl}{Định lý}
\newtheorem{md}{Mệnh đề}
\newtheorem{bd}{Bổ đề}
\newtheorem{hq}{Hệ quả}
\newtheorem{cy}{Chú ý}


\title{Quy hoạch tuyến tính và cơ sở lý thuyết}
\author{Nguyễn Chí Bằng, Đỗ Thư, Lê Đức Anh, Nguyễn Thành Nam} 

\begin{document}

\maketitle

\begin{abstract}
Quy hoạch tuyến tính
\end{abstract}

\tableofcontents

\section{Thuật toán đơn hình}
    \subsection{Thuật toán M và 2 pha}
            \subsubsection{Bài toán M}
                \begin{enumerate}
                    \item \textbf{Giới thiệu} \\
                        Đối với các bài toán quy hoạch tuyến tính nhưng không đủ số vector cơ sở. Ta thêm vào 1 ẩn giả ở ràng buộc và $Mx_4$ vào hàm mục tiêu với $M$ là số dương rất lớn (đối với bài toán Min). Từ đó ta dẫn đến được dạng bài toán quy hoạch tuyến tính M.
                    \item \textbf{Nội dung} \\
                        Giả sử ta có bài toán có ma trận $A$ chưa xác định được hệ cơ sở
                            \begin{equation} \label{*}
                                \begin{split}
                                    Min \: \sum_{j=1}^n c_jx_j \\
                                    \sum_{j=1}^n a_{ij}x_j&=b_i , i = \overline{\rm 1,m} \\
                                    x_j &\geq 0 , j = \overline{\rm 1,n} \\
                                \end{split}
                            \end{equation}
                        Từ đó ta xét BT có ẩn giả tạo sau (gọi là BTM) \eqref{**}
                            \begin{equation} \label{**}
                                \begin{split}
                                    Min \: \sum_{j=1}^n c_jx_j+M \\
                                    \sum_{j=1}^n a_{ij}x_j + x_{n+i}&=b_i , i = \overline{\rm 1,m} \\
                                    x_j &\geq 0 , j= \overline{\rm 1,{n+m}}
                                \end{split}
                            \end{equation}
                        Với M là số thực dương lớn tuỳ ý, các ẩn $x_{n+i} , i=\overline{\rm 1,m}$ là các ẩn giả tạo. Việc đưa các ẩn giả tạo nhằm tạo ra 1 cơ sở. Vì vậy, không nhất thiết phải có đủ $m$ ẩn giả tạo.
                            \begin{dl}
                                Với M đủ lớn thì
                                \item BT \eqref{*}  có PATU X khi và chỉ khi BT \eqref{**} có PATU $\overline{\rm X} = (X,0)$.
                                \item Nếu BT \eqref{**} có PATU $\overline{\rm X}$ mà trong đó có chứa $x_{m+i}>0$ thì tập PATU của BT \eqref{*} là rỗng.
                            \end{dl}
                            \begin{cy}
                                Đối với bài toán M, $\Delta_j$ có dạng $\Delta_j=\alpha_j+\beta_jM$. Vì thế dòng thứ $m+1$ của bảng đơn hình, ta nên tách thành hai dòng ghi $\alpha _j$ và $\beta_j$. Để đánh giá $\Delta_j$ ta cần chú ý tới $B_j$, sau đó mới chú ý đến $\alpha_j$.
                            \end{cy} 
                    \item \textbf{Ví dụ minh hoạ} \\          
                \end{enumerate}
            \subsubsection{Bài toán 2 pha}
                \begin{enumerate}
                    \item \textbf{Nội dung} \\
                        Nếu 1 bài toán quy hoạch tuyến tính có hệ ràng buộc theo dạng
                            \begin{equation}
                                \begin{split}
                                    Ax &\leq b \\
                                    x &\geq c
                                \end{split}
                            \end{equation}
                        Mà $b \geq 0$ thì có thể biến đổi hệ ràng buộc của bài toán dạng chính tắc bằng cách thêm vào hàng thứ $i$ của hệ nói trên ẩn $y_i$. Hệ được viết lại:
                            \begin{equation}
                                \begin{split}
                                    Ax+y&=b\\
                                    x,y &> 0
                                \end{split}
                            \end{equation}
                        Khi đó ta có ngay một phương án xuất phát $(0,\overline{\rm y})$ với $\overline{\rm y_i} = b_i ; i=1,2,\ldots,m$. \\
                        Thực chất các ẩn $y_i$ là các ẩn giả tạo, bởi lẽ trong miền xác định của bài toán không cần đến các biến này. Ý tưởng này được áp dụng cho bài toán dạng chính tăc để tìm phương án cực biên ban đầu như sau: \\
                        Từ bài toán dạng chính tắc
                            \begin{equation} \label{2.1}
                                \begin{split}
                                    Min \: f(x) &= \langle c,x \rangle \\
                                    Ax &= b \\
                                    x &\geq 0
                                \end{split}
                            \end{equation} 
                            \begin{equation} \label{2.2}
                                \begin{split}
                                    \text{Ta xét } Min \: \sum_{i=1}^my \\
                                    Ax+y&=b \\
                                    x,y &\geq 0
                                \end{split}
                            \end{equation}
                        \begin{enumerate}
                            \item[$\blacksquare$] Nếu bài toán \eqref{2.1} có $x_0$ chấp nhận được thì bài toán \eqref{2.2} đạt giá trị tối ưu với $u_0 \: y=0$\\
                            \item[$\blacksquare$] Nếu bài toán \eqref{2.1} ta có $u_0$ chấp nhận được tức là $A_x\neq 0$ với mọi $x \geq 0$ thì bài toán \eqref{2.2} sẽ có $u_0$ chấp nhận được với $y\neq 0$ tức là giá trị tối ưu của bài toán \eqref{2.2} là số dương.
                            \item[$\blacksquare$] Dùng thuật toán đơn hình cho bài toán \eqref{2.2} ở mỗi bước của thuật giải ta nhận được $u_0$ chấp nhận được.
                            \item[$\blacksquare$] Nếu đến bước mà bài toán có giá trị tối ưu $=0$ thì ta sẽ nhận được $u_0$ cơ bản chấp nhận được cuối cùng với các ẩn $y_i=0 , i=1,2,\ldots,m$. \\
                        \end{enumerate}
                        Từ đây, bài toán \eqref{2.1} được giải với $u_0$ cơ sở chấp nhận được vừa tìm.
                    \item \textbf{Ví dụ minh hoạ} \\
                \end{enumerate}
    \subsection{Thuật toán cải biên và đối ngẫu}
            \subsubsection{Phương pháp đơn hình cải biên}
                \begin{enumerate}
                    \item \textbf{Giới thiệu} \\
                        Xét bài toán quy hoạch tuyến tính ở dạng chính tắc
                            \begin{equation}
                                \langle c,x \rangle \rightarrow Max , Ax = b, x\geq 0
                            \end{equation}
                        Trong đó $A$ là ma trận $m \times n$ và giả sử rằng hạng của ma trận $A$ là $m$. Ta đã nghiên cứu phương pháp đơn hình để giải bài toán này. Giả sử ta có phương án cực biên $x=(x_1,x_2,\ldots,x_m,0,\ldots,0)$ với $x_j>0,i=1,\ldots,m$ và cơ sở $A_j, j \in J=\{1,2,\ldots,m \}$ \\
                        Ta đã có công thức:
                            \begin{equation}
                                \begin{split}
                                    A_k&=\sum_{j\in J}z_{jk}A_j \\
                                    \sum_{j \in J}x_jA_j&=b \\
                                    \Delta k &= \sum_{j \in J}z_{jk}c_j-c_k
                                \end{split}
                            \end{equation}
                        Và ta đã chứng minh rằng phương án cực biên $x$ là tối ưu khi và chỉ khi $\Delta_k \geq 0, \forall k \notin J$.
                    \item \textbf{Nội dung} \\
                        Ochard và Hays lợi dụng tính chất và cơ sở bước lặp sau $A_j,j\in J^{'}=J \setminus \{r\} \cup \{s\}$ chỉ khác cơ sở ở bước $A_j,j \in J$ bằng việc thay thế một vector cơ sở, đã đưa ra thuật toán đơn hình cải biên để giám sát khối lượng tính toán và các thông tin cần lưu trữ trong mỗi bước lặp. \\
                        Ta đưa vào các ký hiệu sau: \\
                        $A_j=(A_1,A_2,\ldots,A_m)-$ ma trận cơ sở. \\
                        $\overline{\rm z_k} = (z_{1k},z_{2k},\ldots,z_{mk})^T-$ vector cột, khai triển của $A_k$ theo cơ sở \\
                        $c_j = (c_1,c_2,\ldots,c_m)-$ vector hàng, hệ số hàm mục tiêu ứng với cơ sở. \\
                        $x_j=(x_1,x_2,\ldots,x_m)^T-$ vector cột, các biến cơ sở. \\
                        Sử dụng các ký hiệu đó ta có thể viết: \\
                            \begin{equation} \label{3da}
                                \begin{split}
                                    A_k&=A_j \overline{\rm Z_k} \rightarrow \overline{\rm Z_k}=A_j^{-1}A_k \\
                                    \Delta_k&=c_j\overline{\rm Z_k}-c_k=c_jA_j^{-1}A_k-c_k \\
                                    A_jx_j&=b \rightarrow x_j=A_j^{-1}b
                                \end{split}
                            \end{equation}
                            Như vậy tất cả các đại lượng cần tính toán đều có thể biễu diễn qua ma trận nghịch đảo $A_j^{-1}$. Tuy nhiên ở mỗi bước lặp không cần phải tính nghịch đảo lại toàn bộ ma trận $A_j$, vì nó chỉ khác ma trận nghịch đảo ở bước trước bởi sự thay thế một vector cột. Giả sử ở bước lặp trước ta có $x,J,A_j^{-1}$ và bây giờ ta có $x^{'},J^{'},A_j^{-1}$. Ta xét ma trận đơn vị $V=(v_{ij})_{m \times n}$ \\
                            Ta ký hiệu:
                                \begin{equation}
                                    \begin{split}
                                        Q&=A_j^{-1}=(q_{ij})_{m \times n} \\
                                        Q^{'}&=A_{j^{'}}^{-1}=(q_{ij}^{'})_{m \times n}
                                    \end{split}
                                \end{equation}
                            Ta có:
                                \begin{equation} \label{4}
                                    \begin{split}
                                        A_jA_j^{-1}&=V \\
                                        \Rightarrow \sum_{j \in J}a_{ij}q_{jk}&=v_ik \\
                                        \Rightarrow \sum_{j \in J}q_{jk}A_j&=V_k
                                    \end{split}
                                \end{equation}
                            Tương tự
                            \begin{equation} \label{5}
                                A_j,A_{j^{'}}^{-1}=V \Rightarrow \sum q_{jk}^{'}A_j=V_k
                            \end{equation}  
                            Do $J^{'}=J \setminus \{r\} \cup \{s\}$ và $A_s=\sum_{j \in J}z_{js}A_j$ với $z_{rk}\neq 0$ nên ta có đẳng thức \\
                            $$A_r=\frac{1}{z_{rs}}A_s - \frac{1}{z_{rs}} \sum_{j \in J \setminus \{r\}}z_{js}A_j$$  
                            Thay biểu thức này vào \eqref{4} ta được
                                \begin{equation}
                                    \begin{split}
                                        V_k&=\sum_{j \in J \setminus \{r\}}q_{jk}A_j+(\frac{q_{rk}}{z_{rs}}A_s-\frac{q_{rk}}{z_{rs}}\sum_{j \in J \setminus \{r\}}z_{js}A_j) \\
                                        &=\sum_{j \in J \setminus \{r\}}(q_{jk}-\frac{q_{rk}}{z_{rs}}z_{js})A_j + \frac{q_{rk}}{z_{rs}}A_s
                                    \end{split}
                                \end{equation}
                            Từ hệ thức này và \eqref{5} ta suy ra
                            \begin{equation} \label{6da}
                                q_{jk}^{'}=
                                \begin{cases}
                                    q_{jk}-(\frac{q_{rk}}{z_{rs}})z_{js} \text{ nếu } j \in J \text{ và } j \neq r \\
                                    \frac{q_{rk}}{z_{rs}} \text{ nếu } j = s
                                \end{cases}
                            \end{equation}     
                        \textbf{Thuật toán đơn hình cải biên} \\
                        Xét bài toán QHTT ở dạng chính tắc, quá trình tính toán theo phương pháp đơn hình cải biên được bố trí trong hai bảng sau \\
                            \begin{figure}
                                \caption{Bảng 1}
                                    \begin{center}
                                        \begin{tabular}{|c|c|c|c|c|c|}
                                            \hline
                                            $b_1$ & $a_{11}$ & $\ldots$ & $a_{1s}$ & $\ldots$ & $a_{1n}$ \\
                                            \hline
                                            $\ldots$ & $\ldots$ & $\ldots$ & $\ldots$ & $\ldots$ & $\ldots$ \\
                                            \hline
                                            $b_m$ & $a_{m1}$ & $\ldots$ & $a_{ms}$ & $\ldots$ & $a_{mn}$ \\
                                            \hline
                                            & $c_1$ & $\ldots$ & $c_s$ & $\ldots$ & $c_n$ \\
                                            \hline
                                            $\Delta_{(1)}$ & $\Delta_1$ & $\ldots$ & $\Delta_s$ & $\ldots$ & $\Delta_n$ \\
                                            \hline
                                            $\Delta_{(2)}$ & $\Delta_1$ & $\ldots$ & $\Delta_s$ & $\ldots$ & $\Delta_n$ \\
                                            \hline    
                                        \end{tabular}
                                    \end{center}  
                            \end{figure} 
                            \begin{figure}
                                \caption{Bảng 2}
                                    \begin{center}
                                        \begin{tabular}{|c|c|c|c|c|c|c|}
                                            \hline
                                            $c_1$ & $A_j$ & $q_0$ & $q_1$ & $\ldots$ & $q_m$ & $A_s$ \\
                                            \hline
                                            $c_{j1}$ & $A_{j1}$ & $q_{10}$ & $q_{11}$ & $\ldots$ & $q_{1m}$ & $z_{1s}$ \\
                                            \hline
                                            $\ldots$ & $\ldots$ & $\ldots$ & $\ldots$ & $\ldots$ & $\ldots$ & $\ldots$ \\
                                            \hline
                                            $c_{jr}$ & $A_{jr}$ & $q_{r0}$ & $q_{r1}$ & $\ldots$ & $q_{rm}$ & $z_{rs}$ \\
                                            \hline
                                            $\ldots$ & $\ldots$ & $\ldots$ & $\ldots$ & $\ldots$ & $\ldots$ & $\ldots$ \\
                                            \hline
                                            && $q_{m+1,0}$ & $q_{m+1,1}$ & $\ldots$ & $q_{m+1,m}$ & $\Delta_s$ \\
                                            \hline    
                                        \end{tabular}
                                    \end{center}  
                            \end{figure}  
                        Dòng cuối cùng chứa phương án của bài toán đối ngẫu được tính theo công thức:
                            \begin{equation} \label{7da}
                                (q_{m+1,1},\ldots,q_{m+1,m})=c_jA_j^{-1}
                            \end{equation}
                        \textbf{Thuật toán đơn hình gồm các bước sau:} \\
                            \textbf{Bước 1 (xây dựng bản đơn hình xuất phát)}\\
                                Giả sử ta có cơ sở $A_j,j \in J$ và phương án cực biên $x$. Tính ma trận nghịch đảo $A_j^{-1}$ rồi điền vào $m$ phần tử đầu tiên của các cột $q_1,\ldots,q_m$. Tính $m$ phần tử đầu của cột $q_0$ theo \eqref{3da} và $q_{m+1,0}^{'}=\langle c_j,x_j \rangle$. Tính dòng $m+1$ ứng với các cột $q_1,\ldots,q_m$ theo công thức \eqref{7da} : phần tử $q_{m+1,j}$ là tích vô hướng của cột $q_j$ với vector $c_j$. \\
                            \textbf{Bước 2 (Tìm cột quay và kiểm tra tối ưu)}\\
                                Tính ước lượng các cột theo công thức \eqref{3da} : $\Delta_j$ là tích vô hướng của dòng $m+1$ thuộc bảng 2 với cột $j$ của bảng 1 rồi trừ đi $c_j$. Nếu $\Delta_j \geq 0$ với mọi $j$ thì phương án cực biên đang xét là tối ưu. Trái lại, ta xác định vector $A_s$ đưa vào cơ sở theo công thức
                                    \begin{equation}
                                        \Delta_s = min\{\Delta_j \mid \Delta_j < 0,j \in J \}
                                    \end{equation}
                            \textbf{Bước 3 (tìm dòng quay)}\\
                                Trước tiên tính cột quay, tức là cột $A_s$ của bảng 2 theo công thức \eqref{3da}: lấy cột $A_s$ của bảng 1 nhân vô hướng với từng dòng của ma trận nghịch đảo $A_j^{-1}$ ta sẽ được từng phần tử của cột $A_s$ thuộc bảng 2. Phần từ cuối của cột $A_s$ thuộc bảng 2 là $\Delta_s$\\
                                Nếu $z_{js} \leq 0, \forall j \in J$ thì hàm mục tiêu bài toán QHTT không bị chặn trên. Nếu trái lại ta xác định vector $A_r$ loại khỏi cơ cở theo công thức
                                    \begin{equation}
                                        \theta_r=\frac{q_{r0}}{z_{rs}}=min\{ \frac{q_{j0}}{z_{js}} \mid z_{js} > 0,j \in J \}
                                    \end{equation}
                                Cột $\theta$ trong bảng 2 để lưu $q_{j0}/q_{js},j \in J$.\\
                            \textbf{Bước 4 (Biến đổi ma trận nghịch đảo mở rộng)}\\
                                Đưa $A_s$ vào cơ sở thay ho $A_r$ và biến đổi toàn bộ các cột $q_0,q_1,\ldots,q_m$ theo công thức \eqref{6da} (quy tắc hình chữ nhật), phần tử chính của phép biến đổi biến là $z_{rs}$. Quay lên bước 2. 
                    \item \textbf{Ví dụ minh hoạ} \\
                \end{enumerate}
            \subsubsection{Phương pháp đơn hình đối ngẫu}
                \begin{enumerate}
                    \item \textbf{Giới thiệu} \\
                        Trong thực tế, việc giải một bài toán bằng thuật toán đơn hình ban đầu cho ta mội giải pháp tối ưu nhưng không khả thi. Trong trường hợp này, thuật toán đơn hình đối ngẫu cung cấp cho ta một giải pháp khả thi bằng cách điều chỉnh phương án thông qua $\overline{\rm b}_r$ sao cho $\overline{\rm b}_r \geq 0$ trong khi đó với thuật toán đơn hình, ta tập trung vào $\overline{\rm c}_s$ sao cho $\overline{\rm c}_s < 0$. Trong khi vẫn giữ được tính khả thi.
                    \item \textbf{Nội dung} \\
                        Cho bài toán dạng chính tắc
                        \begin{equation}
                            \begin{split}
                                Min \: c^Tx &= z\\
                                Ax &= b,\\ 
                                x &\geq 0,
                            \end{split}
                        \end{equation}        
                        Như ta thấy, sau quá trình biến đổi với thuật toán đơn hình ta luôn nhận được một hệ phương trình với các biến cơ sở $-z, x_1, x_2, \ldots , x_m$.
                        \begin{center}
                            \begin{tabular} {ccccccccccc}
                                $-z$ &&&&& $+\overline{\rm c}_{m+1}x_{m+1}+$ & $\ldots$ & $+ \overline{\rm c}_jx_j$ & $\ldots$ & $+ \overline{\rm c}_nx_n$ & $= -\overline{\rm z}_0$ \\
                                & $x_1$ &&&& $+\overline{\rm a}_{1,m+1}x_{m+1}+$ & $\ldots$ & $+ \overline{\rm a}_{1j}x_j$ & $\ldots$ & $+ \overline{\rm a}_{1n}x_n$ & $= \overline{\rm b}_1$ \\                            
                                && $x_2$ &&& $+\overline{\rm a}_{2,m+1}x_{m+1}+$ & $\ldots$ & $+ \overline{\rm a}_{2j}x_j$ & $\ldots$ & $+ \overline{\rm a}_{2n}x_n$ & $= \overline{\rm b}_2$ \\                            
                                &&& $\ddots$ && $\vdots$ && $\vdots$ && $\vdots$ & $\vdots$ \\
                                &&&& $x_m$ & $+\overline{\rm a}_{m,m+1}x_{m+1}+$ & $\ldots$ & $+ \overline{\rm a}_{mj}x_j$ & $\ldots$ & $+ \overline{\rm a}_{mn}x_n$ & $= \overline{\rm b}_m,$ \\ 
                            \end{tabular}
                        \end{center}
                        Với $x_B=(x_1,x_2,\ldots,x_m)^T$ và $x_N=(x_{m+1},x_{m+2},\ldots,x_n)^T$, Hệ phương trình được viết lại dưới dạng:
                        \begin{equation}
                            \begin{split}
                                (-z) + \overline{\rm c}x_N &= -\overline{\rm z}_0 \\
                                Ix_B + \overline{\rm A}x_N &= \overline{\rm b},
                            \end{split}
                        \end{equation}     
                Hoặc dưới dạng ma trận, hệ có thể viết lại thành:
                \begin{gather}
                    \begin{pmatrix} 
                        1 & 0 & \overline{\rm c} \\
                        0 & I & \overline{\rm A} \\
                    \end{pmatrix}
                    \begin{pmatrix}
                        -z \\
                        x_{B} \\
                        x_{N} \\
                    \end{pmatrix}
                    =
                    \begin{pmatrix}
                        - \overline{\rm z}_0 \\
                        \overline{\rm b}
                    \end{pmatrix}
                    ,
                \end{gather}
                Từ đây, ta xét dạng đối ngẫu của bài toán
                \begin{equation}
                    \begin{split}
                        Max \: b^T\pi &= v\\
                        A^T\pi & \le c,\\
                    \end{split}
                \end{equation}
                Với các ẩn giả được thêm vào ta viết lại
                \begin{equation}
                    \begin{split}
                        Max \: b^T\pi &= v\\
                        A^T\pi +Iy &= c,\\
                        y \geq 0
                    \end{split}
                \end{equation}
                Xét mối quan hệ giữ bài toán dạng chính tắc và dạng đối ngẫu của nó ta có cái mối tương quan như bảng trên.
                \begin{figure} \label{MQH}
                    \caption{Mối quan hệ giữa bài toán gốc và bài toán đối ngẫu}
                        \begin{center}
                            \begin{tabular} {|c|c|c|}
                                \hline
                                & Bài toán gốc & Bài toán đối 
                                ngẫu \\
                                \hline
                                Cơ sở & $B$ & $\overline{\rm B} =
                                \begin{pmatrix}
                                    B^T & 0 \\
                                    N^T & I_{n-m}
                                \end{pmatrix}$ \\
                                \hline
                                Biến cơ sở & $x_B$ & $\pi,y_N=\overline{\rm c}_N$ \\
                                \hline
                                Biến không cơ sở & $x_N$ & $y_B=\overline{\rm c}_B$ \\
                                \hline
                                Điều kiện & $Ax=b,x \geq 0$ & $\overline{\rm c} \geq 0$ \\
                                \hline
                            \end{tabular}
                        \end{center}
                    \end{figure}

                Ta có thể thấy cột cơ sở và không cơ sở của bài toán dạng đối ngẫu lần lượt được ký hiệu là $\overline{\rm B}$ và $\overline{\rm N}$.
                \begin{gather}
                    \overline{\rm B} =
                        \begin{pmatrix}
                            B^T & 0 \\
                            N^T & I_{n-m}
                        \end{pmatrix}
                        ,
                \end{gather}
                \begin{gather}
                    \overline{\rm N}=
                        \begin{pmatrix}
                            I_m \\
                            0
                        \end{pmatrix}
                        ,
                \end{gather}
                \textbf{Thuật toán đơn hình đối ngẫu gồm các bước sau:} \\
                    \textbf{Bước 1 (Chọn cơ sở vào)} \\
                        Nếu $\overline{\rm b}_r=min \: \overline{\rm b}_i < 0$, chọn $x_r$ tương đương với hàng $r$. \\
                    \textbf{Bước 2 (Chọn cơ sở ra)} \\
                        Nếu $\frac{\overline{\rm c}_s}{-\overline{\rm a}_{rs}}=min \: \frac{\overline{\rm c}_j}{-\overline{\rm a}_{rj}} \geq 0$ trong đó $\overline{\rm a}_{rj} > 0$, loại bỏ $x_s$ tương ứng với cột $s$. \\
                    \textbf{Bước 3 (Kiểm tra)} \\
                        Xoay vòng theo $\overline{\rm a}_{rs}$ để xác định một giải pháp khả thi mới, đặt $j_r=s$ và quay lại bước 1.
                \end{enumerate}
\section{Bài toán đối ngẫu}
    \subsection{Giới thiệu}
        Trong Lý thuyết tối ưu,với một bài toán tối ưu cho trước, người ta quan tâm làm sao thiết lập được bài toán liên kết với bài toán đã cho mà khi giải bài toán này ta thu được thông tin về bài toán ban đầu. Đó chíng là bài toán đối ngẫu. \\
        Việc giải và tìm hiểu hai bài toán song song rất có ý nghĩa về mặt thực tiễn. Đôi khi ta giả bài toán đối ngẫu lại dễ dàng hơn so với bài toán gốc. Vì sao lại dễ dàng hơn thì mục dưới đây chính là trình bày và nghiên cứu bài toán đối ngẫu của bài toán quy hoạch tuyến tính.
    \subsection{Cơ sở lý thuyết}
        \subsubsection{Ví dụ hướng đến bài toán đối ngẫu}
            Một công ty A sản xuất 4 mặt hàng, sử dụng hai loại vật liệu loại $I$ và $II$ với số lượng là $b_1$,$b_2$. Để sản xuất mặt hàng thứ $j$ \: $(j=1,2,3,4)$ cần có $a_{1j}$  đơn vị nguyên liệu loại $I$ và $a_{2j}$ đơn vị nguyên liệu loại $II$. Mặt hàng thứ $j$, được bán tương ứng với giá $cj$. Hãy lập kế hoạch để công ty sản xuất có tổng giá trị sản phẩm cao nhất.
            \begin{center}
                Giải
            \end{center}
            Gọi $x_1$, $x_2$, $x_4$ lần lượt là số lượng mặt hàng cần sản xuất ( $x_i$ $\geq$ 0, $i$ = 1, 2, 3, 4). Bài toán đặt ra là ta cần làm cực đại hàm mục tiêu số lượng như sau: \\
                \begin{equation}
                    f(x)=c_1x_1+c_2x_2+c_3x_3+c_4x_4 \rightarrow Max
                \end{equation}
            Số lượng vật tư $b_1$ được phân bố cho 4 mặt hàng trên : \\
                \begin{equation}
                    a_{11}x_1+a_{12}x_2+a_{13}x_3+a_{14}x_4 \leq b_1
                \end{equation}
            Số lượng vật tư $b_2$ được phân bố cho 4 mặt hàng trên :
                \begin{equation}
                    a_{21}x_1+a_{22}x_2+a_{23}x_3+a_{24}x_4 \leq b_2
                \end{equation}
            Vậy ta thu được thông tin của bài toán với các điều kiện được ghi lại như sau:
                \begin{equation}
                    \begin{split}
                    f(x)=c_1x_1+c_2x_2+c_3x_3+c_4x_4 &\rightarrow Max \\
                    a_{11}x_1+a_{12}x_2+a_{13}x_3+a_{14}x_4 &\leq b_1 \\
                    a_{21}x_1+a_{22}x_2+a_{23}x_3+a_{24}x_4 &\leq b_2 \\
                    x_i &\geq 0 \\
                    i &= 1, 2, 3, 4
                    \end{split}
                \end{equation}
            Dĩ nhiên rằng công ty muốn đầu tư làm sao cho chi phí thấp nhất. Do đó công ty mua vật liệu của công ty khác. Giá bán vật liệu loại $I$ và loại $II$ tương ứng là $y_1$, $y_2$. Công ty phải làm cực tiểu chi phí đầu tư vào, hiển
            nhiên ta có hàm mục tiêu như sau: \\
                \begin{equation}
                    g(y)=b_1y_1+b_2y_2 \rightarrow Min
                \end{equation}
            Ta có thêm các ràng buộc tương ứng về giá cả thương lượng hợp lí của bên bán vật liệu:
                \begin{equation}
                    \begin{split}
                        a_{11}y_1+a_{21}y_2 &\geq c_1 \\
                        a_{12}y_1+a_{22}y_2 &\geq c_2 \\
                        a_{13}y_1+a_{23}y_2 &\geq c_3 \\
                        a_{14}y_1+a_{24}y_2 &\geq c_4 \\
                        y_i &\geq 0 \\
                        i &= 1, 2, 3, 4
                    \end{split}
                \end{equation}
            Hai bài toán trên viết lại dưới dạng ma trận ta sẽ có thông tin ngắn gọn như sau:
                \begin{equation}
                    \begin{split}
                        (P) \: Max f(x) = \langle c,x \rangle &= c^Tx \\
                        (P^*)/(D) \: Min g(y) = \langle b,y \rangle &= b^Ty \\
                        Ax &\leq b \\
                        A^Ty \geq cx &\geq 0 \\
                        y &\geq 0
                    \end{split}
                \end{equation} 
            Hai bài toán trên chính là đối ngẫu của nhau, giải quyết cực đại của bài toán gốc $(P)$ và cực tiểu của bài toán liên kết $(P^*)/(D)$ và tìm phương án tối ưu thông qua đó.
        \subsubsection{Qui tắc đối ngẫu (trang 33 chương 2 quy hoạch tuyến tính)}
            Khảo sát đối ngẫu cho các dạng tổng quát, ta có các qui tắc sau:
                \begin{equation}
                    \begin{split}
                        M &= \{1,2,3, \ldots , m\} , M_1 = \{ 1,2,3, \ldots , m_1 \} , m_1 \leq m \\
                        N &= \{ 1,2,3, \ldots , n\} , N_1 = \{ 1,2,3, \ldots , n_1\} , n_1 \leq n
                    \end{split}
                \end{equation}
            Trường hợp 1: $(P) \: Min \rightarrow (Q) Max$ \\
                \begin{center}
                    \begin{tabular}{|c|c|c|c|}
                        \hline
                        & Gốc $(P)$ & $\Rightarrow$& Đối ngẫu $(Q)$ \\
                        \hline
                        1. & $c^Tx \rightarrow min$ && $b^Ty \rightarrow max$ \\
                        \hline
                        2. &  $\sum_{j=1}^n a_{ij} \geq b_i$ & $i \in M_1$ & $y_i \geq  0 , i \in M_1$ \\
                        \hline
                        3. & $\sum_{j=1}^n a_{ij} = b_i$ & $i \in M \backslash M_1$ & $y_i$ tự do , $i \in M\backslash M_1$ \\
                        \hline
                        4. & $x_j \geq 0$ & $j \in N_1$ & $\sum_{i=1}^m a_{ij}y_i \leq c_j , j \in N_1$ \\
                        \hline
                        5. & $x_j$ có dấu tuỳ ý & $j \in N_1$ & $\sum_{i=1}^m a_{ij} y_i = c_j , j \in N\backslash N_1$ \\
                        \hline
                        6. & $x_j \leq 0$ & $j \in N_2$ & $\sum_{i=1}^m a_{ij}y_i \geq c_j , j \in N_2$ \\
                        \hline
                    \end{tabular}
                \end{center}
            Chú ý: Nếu có $\sum_{j=1}^n \leq b_i$, thì theo nguyên tắc đối ngẫu sẽ ứng với biến $y_i \leq 0$. Trong thực tế nên chuyển về trường hợp 2 bằng cách nhân cả hai vế của bđt cho $-1$. \\
            \vspace{20pt}
            \\            
            Trường hợp 2: $(P) \: Max \rightarrow (Q) Min$ \\
                \begin{center}
                    \begin{tabular}{|c|c|c|c|}
                        \hline
                        & Gốc $(P)$ & $\Rightarrow$ & Đối ngẫu $(Q)$ \\
                        \hline
                        1. & $c^Tx \rightarrow max$ && $b^Ty \rightarrow min$ \\
                        \hline
                        2. &  $\sum_{j=1}^n a_{ij} \leq b_i$ & $i \in M_1$ & $y_i$ tự do, $i \in M_1$ \\
                        \hline
                        3. & $\sum_{j=1}^n a_{ij} = b_i$ & $i \in M \backslash M_1$ & $y_i$ tự do , $i \in M\backslash M_1$ \\
                        \hline
                        4. & $x_j \geq 0$ & $j \in N_1$ & $\sum_{i=1}^m a_{ij}y_i \geq c_j , j \in N_1$ \\
                        \hline
                        5. & $x_j$ có dấu tuỳ ý & $j \in N_1$ & $\sum_{i=1}^m a_{ij} y_i = c_j , j \in N\backslash N_1$ \\
                        \hline
                        6. & $x_j \leq 0$ & $j \in N_2$ & $\sum_{i=1}^m a_{ij}y_i \leq c_j , j \in N_2$ \\
                        \hline
                    \end{tabular}
                \end{center}
            Chú ý: Nếu có $\sum_{j=1}^n a_{ij} \geq b_i$, thì theo nguyên tắc đối ngẫu sẽ ứng với biến $y_i \leq 0$. Trong thực tế nên chuyển về trường hợp 2 bằng cách nhân cả hai vế của bất đẳng thức cho $-1$. (Các ràng buộc ẩn tự do thường ta không cần ghi và bỏ qua nó).
        \subsubsection{Bài toán đối ngẫu dạng chuẩn tắc}
            Bài toán chuẩn tắc có dạng như sau, với $A \in M_{m \times n},b \in R^m, c \in R^n$ \\ 
                \begin{equation}
                    \begin{split}
                        (P) \: c^Tx &\rightarrow Min \\
                        Ax &\geq b \\
                        x &\geq 0 
                    \end{split}
                \end{equation}
            Với bài toán như trên ta có bài toán đối ngẫu như sau với $y \in R^m$ : \\
                \begin{equation}
                    \begin{split}
                        (D) \: b^Ty &\rightarrow Max \\
                        A^Ty &\leq c \\
                        y &\geq 0
                    \end{split}
                \end{equation}
            Cặp bài toán trên được gọi là cặp bài toán đối ngẫu đối xứng vì chứa đủ ba phần : Hàm mục tiêu,ràng buộc đẳng thức (bất đẳng thức) và biến không âm. \\
                \begin{dl}
                    Với mọi phương án $x$ của bài toán $(P)$ và mọi phương án y của bài toán $(P^*)/(D)$, ta có $$g(y) \leq f(x)$$
                \end{dl}
                \begin{dl}
                    \item Nếu bài toán $(P)$ có phương án tối ưu thì bài toán $(P^*)/(D)$ cũng có phương án tối ưu và ngược lại. Đồng thời $V(P)=V(P^*)$.
                    \item Nếu hàm mục tiêu của bài toán này không bị chặn thì tập các phương án của bài toán kia là rỗng.
                \end{dl}
                \begin{hq}
                    Nếu các bài toán $(P)$ và $(P^*)$ đều có các phương án khác rỗng thì chúng đều có phương án tối ưu.
                \end{hq}
                \begin{hq}
                    Giả sử $x^*$ và $y^*$ lần lượt là các phương án tối ưu tương ứng của bài toán $(P)$ và $(P^*)$. Ta có:
                    $$V(P)=f(x^*)=g(y^*)=V(P^*)$$
                \end{hq}
                \begin{hq}
                    Giả sử $x^*$ và $y^*$ tương ứng là các phương án chấp nhận được của các bài toán $(P)$ và $(P^*)/(D)$. Nếu $f(x^*)=g(y^*)$ thì $x^*$ và $y^*$ lần lượt là phương án tối ưu của $(P)$ và $(P^*)/(D)$.
                \end{hq}
                \begin{dl}{\textbf{Độ lệch bù trong đối ngẫu đối xứng}}\\
                    Cho $x^*$ và $y^*$ tương ứng là phương án của bài toán $(P)$ và $(P^*)$. Phương án $x^*$ và $y^*$ tối ưu khi và chỉ khi:
                    $$x_j^* > 0 \Rightarrow \sum_{i=1}^n a_{ij}y_i^*=c_j \text{ hoặc}$$
                    $$\sum_{i=1}^n a_{ij}y_j^*<c_j \Rightarrow x_j^* = 0 \text{ với } j=\overline{\rm 1,n}$$
                    $$y_j^* > 0 \Rightarrow \sum_{i=1}^n a_{ij}x_i^*=b_j \text{ hoặc}$$
                \end{dl}
        \subsubsection{Bài toán đối ngẫu dạng chính tắc}
            Bài toán chính tắc có dạng như sau, với $A \in M_{m \times n},b \in R^m, c \in R^n$:
                \begin{equation}
                    \begin{split}
                        (P) \: c^Tx &\rightarrow Min \\
                        Ax&=b \\
                        x &\geq 0
                    \end{split}
                \end{equation}
            Với bài toán như trên ta có bài toán đối ngẫu như sau với $y \in R^m$:
                \begin{equation}
                    \begin{split}
                        (D) \: b^Ty &\rightarrow Max \\
                        A^Ty &\leq c
                    \end{split}
                \end{equation}
            "Bài toán đối ngẫu không đối xứng như trên có tất cả cá tính chất như đối ngẫu đối xứng". \\
            Định lí về độ lệch bù trong đối ngẫu không đối xứng (hay còn gọi là độ lệch bù yếu).
            \begin{dl}
                Cặp $(x^*,y^*)$ tướng ứng là nghiệm tối ưu của bài toán $(1)$ và $(2)$ khi và chỉ khi
                $$\text{Nếu } x_j^* > 0 \text{ thì } \sum_{i=1}^ma_{ij}y_i^*=c_j \text{ hoặc}$$ \\
                $$\text{Nếu } \sum_{i=1}^ma_{ij}y_i^* < c_j \text{ thì } x_j^*=0$$
            \end{dl}
    \subsection{Ví dụ minh hoạ}

\section{Tài liệu tham khảo}

\bibliographystyle{apa}
\bibliography{references}

\end{document}