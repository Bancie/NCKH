\documentclass{beamer}
\mode<presentation>
\usepackage{beamerthemesplit}
\usepackage[utf8]{vietnam} 
\usepackage{amsmath}
\usepackage{amsfonts}
\usepackage{amssymb}

\usepackage{textpos}

\usepackage{graphicx}

\newtheorem{dn}{Định nghĩa}[section]
\newtheorem{dl}{Định lý}[section]
\newtheorem{tc}{Tính chất}[section]
\newtheorem{hq}{Hệ quả}[section]
\newtheorem{bd}{Bổ đề}[section]
\newtheorem{md}{Mệnh đề}[section]
\newtheorem{vd}{Ví dụ}[section]
\newtheorem{nx}{Nhận xét}[section]
\newcommand{\dom}{\text{{\rm dom}}}
\newcommand{\epi}{\text{{\rm epi}}}
\newcommand{\Min}{\text{{\rm Min}}}
\setbeamertemplate{theorems}[numbered]
\setbeamertemplate{definitions}[numbered]
\setbeamertemplate{footline}[frame number]
%thêm
\usepackage[utf8]{vietnam}
\usepackage{algorithm}
\usepackage{color}
\usepackage{algorithmic}
\usepackage{footmisc}
%\usepackage{enumitem}
\usepackage{indentfirst} 
\usepackage{comment}
\renewcommand{\thefootnote}{\arabic{footnote}}
\usefonttheme{professionalfonts}
\setbeamercolor{normal text}{bg=white,fg=black}
\renewcommand{\thefootnote}{\arabic{footnote}}
%kt
\mode<presentation>
{
 \usetheme{Darmstadt}
%\usetheme{Rochester}
}
\beamertemplatetransparentcoveredhigh
\begin{document}
\title[]{\fontsize{13pt}{10pt}\selectfont {\bf \LARGE   Phương pháp giải bài toán \\Tối ưu tuyến tính nguyên}\\
------------------------------------------

{\small Hướng dẫn: PGS.TS. Tạ Quang Sơn}} 
\author[]{\bf Thực hiện : ĐỖ NGỌC MINH THƯ \& NGUYỄN CHÍ BẰNG \\
Sinh viên lớp: DTU1221, Khóa: 22}
\institute[Báo cáo luận văn thạc sĩ]{\fontsize{2pt}{2pt}}%
\small{\date{\today}}
\begin{frame}
\begin{center}
{\fontsize{8pt}{8pt}\selectfont \bf{ỦY BAN NHÂN DÂN THÀNH PHỐ HỒ CHÍ MINH\\
TRƯỜNG ĐẠI HỌC SÀI GÒN}}
\end{center}
\begin{center}
\end{center}

\begin{center}
{\fontsize{10pt}{6pt}\selectfont \bf{BÁO CÁO ĐỀ CƯƠNG NGHIÊN CỨU KHOA HỌC\\
NGÀNH: TOÁN ỨNG DỤNG}}
\end{center}
\titlepage
\end{frame}





\frame{

\tableofcontents

}

\section{Lý do chọn đề tài }\label{ld}
\transblindshorizontal 
\frame
{
  \frametitle{\bf \ref{ld}. Lý do chọn đề tài}
• Như đã biết, nếu $f\colon \mathbb{R}^n \to \mathbb{R}\cup \{+\infty\}$ là một hàm lồi và khả vi trên ${\rm dom}f$ thì tại điểm $z \in {\rm dom}f$, bất đẳng thức sau đây xảy ra:
\begin{equation}\label{ct1}
 f(x) \ge f(z) + \langle \nabla f(z), x-z \rangle, \forall x \in {\rm dom}f,
 \end{equation}
 trong đó $\nabla f(z)$ là véc tơ các đạo hàm riêng của $f$ tại $z$. Một câu hỏi rất tự nhiên được đặt ra đó là:  Bất đẳng thức \eqref{ct1} sẽ thế nào nếu hàm lồi $f$ không khả vi? 
 
• Câu hỏi được trả lời rằng (xem \cite[Trang 84]{Ba}), nếu $f$ là hàm lồi trên $\mathbb{R}^n$  và không khả vi thì với mọi $z \in {\rm int(dom}f)$ luôn  tồn tại véc tơ $u \in \mathbb{R}^n$ sao cho 
$$
 f(x) \ge f(z) + \langle u, x-z \rangle, \forall x \in {\rm dom}f.
$$
 Véc tơ $u$ nêu trên nói chung là không duy nhất. Trường hợp hàm lồi $f$ là khả vi thì véc tơ $u$ chính là $\nabla f(z)$. Các nghiên cứu cũng chỉ ra rằng tập các véc tơ $u$ nói trên, ký hiệu $\partial f(z)$ là các tập lồi, đóng.
 
• Ta xét bài toán tối ưu lồi sau:
$$
\begin{array}{cl}
{\rm Minimize}& f(x)\\
{\rm s.t.}&x \in C,
\end{array}
$$
}
\transblindshorizontal 
\frame
{
  \frametitle{\bf \ref{ld} Lý do chọn đề tài}
trong đó các hàm số $f$ là hàm lồi xác định trên $\mathbb{R}^n$ và $C$ là tập lồi đóng trong $\mathbb{R}^n$. 
Điều kiện cần và đủ  cho nghiệm tối ưu $z$ của bài toán là
$$0 \in \partial f(z)+ N_C(z). $$

• Mặc dù Lý thuyết về dưới vi phân lồi đã có những đóng góp quan trọng cho nhiều ngành Toán học, đặc biệt về Lý thuyết tối ưu, nhưng vẫn còn những hạn chế cần phải cải tiến. Chẳng hạn, nếu $f(z)=+\infty$ thì $\partial f(z)=\emptyset.$ Vì thế, việc tìm kiếm các dạng dưới vi phân của các hàm không lồi vẫn thu hút sự quan tâm của nhiều nhà nghiên cứu. 

• Ngoài dưới vi phân lồi và dưới vi phân Clarke, có thể liệt kê thêm ở đây một số dạng dưới vi phân suy rộng được quan tâm như: Dưới vi phân Michel-Penot, Dưới vi phân Fr\'echet, Dưới vi phân Dini,\ldots,  và mới gần đây có thêm dưới vi phân yếu Gazimov. Lý thuyết về dưới vi phân là nền tảng để nghiên cứu về các điều kiện tối ưu trong Lý thuyết tối ưu.  Vì vậy, việc tìm hiểu về dưới vi phân là vấn đề rất có ý nghĩa.
}
\section{Mục đích của luận văn}\label{md}
\transblindshorizontal 
\frame
{
  \frametitle{\bf \ref{md}. Mục đích của luận văn}
 •	 Mục đích của nghiên cứu này là thực hiện một sự khảo cứu có tính lý thuyết về một số dạng dưới vi phân của các hàm không trơn, hướng đến việc phục vụ cho việc nghiên cứu các điều kiện tối ưu của một số  lớp các bài toán tối ưu không trơn.
 }
  
 \section{Dự kiến nội dung của luận văn}\label{tc}
 \transblindshorizontal 
\frame{
\frametitle{\bf \ref{tc}. Dự kiến nội dung của luận văn}
Ngoài phần mở đầu, kết luận và danh mục các tài liệu tham khảo, luận văn sẽ được tổ chức thành 3 chương. Trong đó: 


\medskip

$\bullet$ Chương 1: Bao gồm các kiến thức chuẩn bị, nội dung có liên quan đến một số kiến thức cơ bản của Giải tích lồi và Giải tích Lipschitz để dùng làm cơ sở nghiên cứu về các dạng dưới vi phân và cách thiết lập điều kiện tối ưu dạng KKT cho các bài toán tối ưu.


$\bullet$ Chương 2: Tìm hiểu các dạng dưới vi phân lồi, dưới vi phân Clarke, dưới vi phân Mordukhovich và các vấn đề có liên quan

$\bullet$ Chương 3: Tìm hiểu về lý thuyết các điều kiện tối ưu. Tìm hiểu cách thiết lập các điều kiện cần và đủ tối ưu cho một số lớp bài toán lồi và lồi suy rộng dựa trên các dạng dưới vi phân.
 }
 
 \section{Dự kiến tiến độ thực hiện luận văn}\label{tc}
 \transblindshorizontal 
\frame{
\frametitle{\bf \ref{tc}. Dự kiến tiến độ thực hiện luận văn}

Thời gian nghiên cứu chia làm 3 giai đoạn:

$\bullet$ Giai đoạn 1 (2 tháng): Đọc, hiểu tài liệu liên quan đến lý thuyết dưới vi phân và lý thuyết về các điều kiện tối ưu liên quan các lớp bài toán toán lồi và lồi suy rộng. 

$\bullet$Giai đoạn 2 (3 tháng): Thu hoạch, hệ thống lại các tri thức và viết luận văn.


$\bullet$Giai đoạn 3 (1 tháng): Hoàn thành và bảo vệ luận văn.

}

 \section{Giới thiệu sơ lược vấn đề nghiên cứu}\label{gt}
 \transblindshorizontal 
\frame{
\frametitle{\bf \ref{gt}. Giới thiệu sơ lược vấn đề nghiên cứu}
	
	Xét bài toán tối ưu tổng quát dạng như sau:
\[\begin{gathered}
  (P)\,\,\,\,\,\,\Min \,f(x) \hfill \\
  \,\,\,\,\,\,\,\,\,\,\,\,\,\left\{ \begin{gathered}
  {g_i}\left( x \right){\text{ }} \leqslant {\text{ }}0,{\text{ }}i{\text{ }} = {\text{ }}1,{\text{ }}2,{\text{ }} \ldots ,{\text{ }}m. \hfill \\
  x \in S. \hfill \\ 
\end{gathered}  \right. \hfill \\ 
\end{gathered} \]

Trong đó $f,{\text{ }}{g_i}{\text{,}}:{\mathbb{R}^n}{\text{ }} \to {\text{ }}\mathbb{R},{\text{ i }} = {\text{ }}1,{\text{ }}2,{\text{ }}...,{\text{ }}m$, $S$ là tập con của ${\mathbb{R}^n}$. 
	  }
	  
	  

\begin{frame}[allowframebreaks]

\section{Tài liệu tham khảo}\label{tltk}
  \frametitle{\bf \ref{tltk}. Tài liệu tham khảo}
\begin{thebibliography}{xxx}
%1%
\bibitem{Win}W. Schirotzek (2007), \textit {Nonsmooth Analysis}, Springer-Verlag, Berlin-Heidelberg.
%2%
\bibitem{Penot} J-P Penot (2013), \textit {Calculus without Derivatives}, Springer, Newyork.
\bibitem{Bo} J.M. Borwein, Xianfu Wang (1997), \textit {Distinct differentiable functions may share the same Clarke subdifferential at all points}, Proceeding of the American Mathematical Society.
\bibitem{Ba}M.S. Barazaa, C.M. Shetty (1990), \textit{Nonlinear Programming: Theory and Algorithms}, John Wiley and Sons, Singapore.
\bibitem{Cl} Clarke F.H (1983), \textit {Optimization and nonsmooth analysis}, Willey-Interscience, New York.
\bibitem {Rock} R.T. Rockafellar (1972), \textit {Convex Analysis}, Princeton University Press.

\bibitem{tqsongtl} Tạ Quang Sơn (2017), \textit{Giáo trình Giải tích lồi và Tối ưu}, nhà xuất bản Giáo dục, Việt Nam.
\bibitem {Nb} Nguyễn Xuân Tấn, Nguyễn Bá Minh (2008), \textit {Lý thuyết Tối ưu không trơn}, Nhà xuất bản Đại học quốc gia Hà Nội.
\end{thebibliography}
\end{frame}
\begin{frame}\frametitle{Kết thúc}
\begin{block}{}
\medskip
\center{\huge \it \textcolor[rgb]{0.50,0.30,1.0}{Cảm ơn quý thầy cô và các anh chị đã quan tâm theo dõi!}}
\medskip
\end{block}	
\end{frame}
\end{document}