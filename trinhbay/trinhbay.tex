\documentclass{beamer}
\setbeamertemplate{bibliography item}{}
\usepackage[utf8]{vietnam}
\usepackage{graphicx}
\usepackage{booktabs}
\setbeamertemplate{bibliography item}[text]
\usetheme{Warsaw}
\newtheorem{dn}{Định nghĩa}[section]
\newtheorem{dl}{Định lý}[section]
\newtheorem{tc}{Tính chất}[section]
\newtheorem{hq}{Hệ quả}[section]
\newtheorem{bd}{Bổ đề}[section]
\newtheorem{md}{Mệnh đề}[section]
\newtheorem{vd}{Ví dụ}[section]
\newtheorem{nx}{Nhận xét}[section]
\newcommand{\dom}{\text{{\rm dom}}}
\newcommand{\epi}{\text{{\rm epi}}}
\newcommand{\Min}{\text{{\rm Min}}}
\setbeamertemplate{theorems}[numbered]
\setbeamertemplate{definitions}[numbered]
\setbeamertemplate{footline}[frame number]
\usepackage{algorithm}
\usepackage{color}
\usepackage{algorithmic}
\usepackage{footmisc}
%\usepackage{enumitem}
\usepackage{indentfirst} 
\usepackage{comment}
\renewcommand{\thefootnote}{\arabic{footnote}}
\usefonttheme{professionalfonts}
\setbeamercolor{normal text}{bg=white,fg=black}
\renewcommand{\thefootnote}{\arabic{footnote}}
%kt
\mode<presentation>
{
 \usetheme{Darmstadt}
%\usetheme{Rochester}
}
\beamertemplatetransparentcoveredhigh

\begin{document}
\title[]{\fontsize{13pt}{10pt}\selectfont {\bf \LARGE Phương pháp nhánh cận}\\}
\author[]{\bf Nguyễn Chí Bằng \\}
\small{\date{\today}}
\begin{frame}
\begin{center}
\end{center}
\titlepage
\end{frame}

\begin{frame}
    \frametitle{NỘI DUNG}
    \tableofcontents
\end{frame}

\section{Quy hoạch nguyên}
\subsection{Quy hoạch nguyên hoàn toàn}
\begin{frame}
    \begin{center}                    
        \big(H\big)\quad $f(x)=cx \quad \longrightarrow Max$ \\
        \[\left\{\begin{aligned}
            &Ax \leq  b \\
            &x\geq 0, \text{ nguyên}
        \end{aligned}\right.\]\\
    \end{center}
    \begin{itemize}
    \item Trong đó $c=(c_1,\ldots,c_n)$, $A$ là ma trận $m\times n$, $b^T=(b_1,\ldots,b_m)$, và $x$ là $n-\text{vector}$ với $x\in Z^n$.
    \item Bài toán $(H)$ gọi là bài toán \textit{quy hoạch nguyên hoàn toàn.}
    \item Tập $S_h:=\{x\in Z^n_+: Ax\leq b\}$ là tập nghiệm của bài toán quy hoạch nguyên hoàn toàn.
    \end{itemize}
\end{frame}
\subsection{Quy hoạch nguyên bộ phận}
\begin{frame}
    \begin{center}                    
        \big(B\big)\quad $f(x)=cx+hy \quad \longrightarrow Max$ \\
        \[\left\{\begin{aligned}
            &Ax+Gy \leq  b \\
            &x\geq 0, \text{ nguyên} \\
            &y\geq 0,
        \end{aligned}\right.\]\\
    \end{center}
    \begin{itemize}
    \item Trong đó $c=(c_1,\ldots,c_n)$, $h=(c_{n+1},\ldots,h_p)$, $A$ là ma trận $m\times n$, $G$ là ma trận $m\times p$, $b^T=(b_1,\ldots,b_m)$, $x, y$ là $n-\text{vector}$ với $x\in Z^n$ và $y\in R^p$.
    \item Bài toán $(B)$ gọi là bài toán \textit{quy hoạch nguyên hoàn toàn.}
    \item Tập $S_b:=\{(x,y)\in Z^n_+\times R^p_+: Ax+Gy\leq b\}$ là tập nghiệm của bài toán quy hoạch nguyên bộ phận.
    \end{itemize}
\end{frame}
\section{Phương pháp nhánh cận}
\subsection{Mở rộng ràng buộc}
\begin{frame}
    Ta sẽ xử lý bài toán $(B)$ thông qua bài toán $(B_0)$ sau:
    \begin{center}                    
        \big($B_0$\big)\quad $f(x)=cx+hy \quad \longrightarrow Max$ \\
        \[\left\{\begin{aligned}
            &Ax+Gy \leq  b \\
            &x,y \geq 0,
        \end{aligned}\right.\]\\
    \end{center}
\begin{itemize}
\item Tập $S_0:=\{(x,y)\in R^n_+\times R^p_+: Ax+Gy\leq b\}$ là tập nghiệm của bài toán $(B_0)$
\end{itemize}
\end{frame}
\begin{frame}
    
\end{frame}
\end{document}