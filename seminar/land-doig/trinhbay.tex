\documentclass{beamer}
\mode<presentation>
\setbeamertemplate{bibliography item}{}
\usepackage[utf8]{vietnam}
\usepackage{beamerthemesplit}
\usepackage{graphicx}
\usepackage{booktabs}
\usepackage{amsmath}
\usepackage{textpos}
\usepackage{pgfplots}
\usepackage{tikz}
\usepackage{hyperref}
\usepackage{caption}

\usetikzlibrary {datavisualization} 
\pgfplotsset{compat=1.18, width = 7cm}
\usetikzlibrary{patterns}
\setbeamertemplate{bibliography item}[text]
\usetheme{Ilmenau} % AnnArbor, Ilmenau, Darmstadt, Dresden, CambridgeUS, Frankfurt, Singapore
\newtheorem{dn}{Định nghĩa}[section]
\newtheorem{dl}{Định lý}[section]
\newtheorem{tc}{Tính chất}[section]
\newtheorem{hq}{Hệ quả}[section]
\newtheorem{bd}{Bổ đề}[section]
\newtheorem{md}{Mệnh đề}[section]
\newtheorem{vd}{Ví dụ}[section]
\newtheorem{nx}{Nhận xét}[section]
\newcommand{\dom}{\text{{\rm dom}}}
\newcommand{\epi}{\text{{\rm epi}}}
\newcommand{\Min}{\text{{\rm Min}}}
\setbeamertemplate{theorems}[numbered]
\setbeamertemplate{definitions}[numbered]
\setbeamertemplate{footline}[frame number]
\usepackage{algorithm}
\usepackage{color}
\usepackage{algorithmic}
\usepackage{footmisc}
\usepackage{indentfirst} 
\usepackage{comment}
\renewcommand{\thefootnote}{\arabic{footnote}}
\usefonttheme{professionalfonts}
\setbeamercolor{normal text}{bg=white,fg=black}
\renewcommand{\thefootnote}{\arabic{footnote}}
\beamertemplatetransparentcoveredhigh
\title[]{\fontsize{13pt}{10pt}\selectfont {\bf \LARGE Phương pháp nhánh cận}\\}
\author[]{\bf Nguyễn Chí Bằng \\}
\small{\date{\today}}

\begin{document}

\begin{frame}

\titlepage
\end{frame}

\begin{frame}{TÓM TẮT}
\begin{itemize}
\item Giới thiệu về 2 mô hình chính trong phương pháp giải bài toán tối ưu tuyến tính nguyên:
\begin{itemize}
\item Tối ưu nguyên hoàn toàn
\item Tối ưu nguyên bộ phận.
\end{itemize}
\item Tập trung vào phương pháp giải bài toán tối ưu nguyên bộ phận thông qua thuật toán nhánh cận.
\end{itemize}
\end{frame}

\begin{frame}
    \frametitle{NỘI DUNG}
    \tableofcontents
\end{frame}

\section{Giới thiệu}

\begin{frame}
   \center 
   \huge Giới thiệu bài toán 
\end{frame}

\begin{frame}{Tối ưu nguyên hoàn toàn (Pure integer linear program)}
    \begin{equation} \label{H}
        \begin{split}
        (H) \quad & z_h=c^Tx \quad \longrightarrow Max \\
                  & \left\{\begin{split}
                    &Ax \leq  b, \\
                    &x \geq 0, \text{ nguyên} \\
                    \end{split}\right.    
        \end{split}
        \end{equation}            
    \begin{itemize} \small
    \item Trong đó $c^T=(c_1 \: c_2 \: \ldots \: c_n)$, $A$ là ma trận $m\times n$, $b=\begin{pmatrix}
        b_1 \\
        b_2 \\
        \vdots \\
        b_m
        \end{pmatrix}$, với $x\in Z^n$.
    \item Bài toán $(H)$ gọi là bài toán \textbf{Tối ưu nguyên hoàn toàn.}
    \item Tập $S_h:=\{x\in Z^n_+: Ax\leq b\}$ là tập nghiệm của bài toán Tối ưu nguyên hoàn toàn.
    \end{itemize}
\end{frame}

%model_1: hoan toan
\begin{frame} {Minh hoạ bài toán}
    \begin{equation}
        \begin{split}
        \quad & 2x_1 + 2x_2 \quad \longrightarrow Max \\
                    & \left\{\begin{split}
                    & x_1 + 3x_2 \leq 24 \\
                    & \frac{13}{3}x_1 + 2x_2 \leq 32.5 \\
                    &x_1, x_2 \geq 0. \\
                    \end{split}\right.    
        \end{split}
        \end{equation}            
\end{frame}
%cho bài toán ví dụ rõ số liệu
\begin{frame}
\begin{center}
\begin{figure}
\begin{tikzpicture}
\begin{axis}
    [
    xmin=0,xmax=10,
    ymin=0,ymax=10,
    grid=both,
    grid style={line width=.1pt, draw=darkgray!50},
    major grid style={line width=.2pt,draw=darkgray!50},
    axis lines=middle,
    minor tick num=1,
    enlargelimits={abs=0},
    samples=100,
    domain = -20:20,
    ]
    \filldraw[green, pattern=north west lines, pattern color=green, line width=2pt] (0, 0) -- (0, 8) -- (4.5, 6.5) -- (7.5, 0) -- cycle;
\end{axis}
\end{tikzpicture}  
\caption{Tập nghiệm của bài toán Tối ưu nguyên hoàn toàn}
\end{figure}
\end{center}      
\end{frame}

\begin{frame}{Tối ưu nguyên bộ phận (Mixed integer linear program)}
\begin{equation}\label{B}
\begin{split}
(B) \quad & z_b=c^Tx+h^Ty \quad \longrightarrow Max \\
          & \left\{\begin{split}
            &Ax+Gy \leq  b, \\
            &x \geq 0, \text{ nguyên} \\
            &y \geq 0.
            \end{split}\right.    
\end{split}
\end{equation}    
\begin{itemize} \small
\item Trong đó $c^T=(c_1 \: c_2 \: \ldots \: c_n)$, $h^T=(h_1 \: h_2 \: \ldots \: h_p)$, $A$ là ma trận $m\times n$, $G$ là ma trận $m\times p$, $b=\begin{pmatrix}
    b_1 \\
    b_2 \\
    \vdots \\
    b_m
    \end{pmatrix}$, với $x\in Z^n$ và $y\in R^p$.
\item Bài toán $(B)$ gọi là bài toán \textbf{Tối ưu nguyên bộ phận.}
\item Tập $S_b:=\{(x,y)\in Z^n_+\times R^p_+: Ax+Gy\leq b\}$ là tập nghiệm của bài toán Tối ưu nguyên bộ phận.
\end{itemize}
\end{frame}
%cho bài toán ví dụ rõ số liệu
\begin{frame}{Minh hoạ bài toán}
   \begin{equation}
        \begin{split}
        \quad & x_1 + 2x_2 \quad \longrightarrow Max \\
                    & \left\{\begin{split}
                    & 5x_1 + \frac{15}{7}x_2 \leq 20 \\
                    & -2.4x_1 + \frac{30}{7}x_2 \leq 15 \\
                    &x_1, x_2 \geq 0. \\
                    \end{split}\right.    
        \end{split}
        \end{equation}            
\end{frame}

\begin{frame}
\begin{center}
\begin{figure}
\begin{tikzpicture}
\begin{axis}
    [
    xmin=0,xmax=5,
    ymin=0,ymax=5,
    grid style={line width=.1pt, draw=darkgray!50},
    major grid style={line width=.2pt,draw=darkgray!50},
    axis lines=middle,
    minor tick num=1,
    enlargelimits={abs=0},
    samples=100,
    domain = -20:20,
    xmajorgrids=true,
    ]
    \filldraw[green, pattern=north west lines, pattern color=green, line width=2pt] (0, 0) -- (0, 2.5) -- (2.5, 3.5) -- (4, 0) -- cycle;
\end{axis}
\end{tikzpicture}  
\caption{Tập nghiệm của bài toán Tối ưu nguyên bộ phận}
\end{figure}
\end{center}      
\end{frame}
    
\section{Mục tiêu}

\begin{frame}
   \center
   \huge Mục tiêu phương pháp
\end{frame}

\begin{frame}{Bài toán quan tâm}
\begin{equation}\label{P}
\begin{split}
(P) \quad & z_p=c^Tx+h^Ty \quad \longrightarrow Max \\
            & \left\{\begin{split}
                &Ax+Gy \leq  b, \\
                &x,y \geq 0.
            \end{split}\right.    
\end{split}
\end{equation}
\begin{itemize}
\item Trong đó $(P)$ là bài toán $(B)$ với nghiệm thuộc tập số thực.
\item Bài toán $(P)$ là một bài toán \textbf{Tối ưu tuyến tính thông thường} hay gọi đơn giản là bài toán Tối ưu tuyến tính (Natural linear programming relaxation).
\item Tập $S_p:=\{(x,y)\in R^n_+\times R^p_+: Ax+Gy\leq b\}$ là tập nghiệm của bài toán Tối ưu tuyến tính.
\end{itemize}
\end{frame}

\begin{frame}{Mục tiêu} %aka Lí do chọn bài toán
Giả sử ta nhận được tập phương án tối ưu của bài toán \eqref{B} sau hữu hạn lần giải, ký hiệu $(x_b, y_b)$ và giá trị tối ưu là $z_b$ thì ta có nhận xét sau:
\begin{nx}
\begin{itemize}
\item Nếu $S_b \subset S_p$ thì ta luôn nhận được $z_b < z_p$ và phương án có thể cải thiện.
\item Nếu $S_b = S_p$ thì ta nhận được $z_b = z_p$ và bài toán được giải.
\end{itemize}    
\end{nx}
Vì thế, ta chọn xử lý bài toán $(B)$ thông qua bài toán $(P)$ bằng cách cải thiện phương án thu được từ bài toán $(P)$ sao cho thoả điều kiện của bài toán $(B)$.
\end{frame}

\begin{frame}{Ví dụ}
    \begin{columns}
    \begin{column}{0.1\textwidth}
    \end{column}
    \begin{column}{0.4\textwidth}
        \begin{equation*}
        \begin{split}
            (P) \quad & 5.5x_1 + 2.1x_2 \quad \longrightarrow Max \\
            & \left\{\begin{split}
            & -1x_1 + x_2 \leq 2 \\
            & 8x_1 + 2x_2 \leq 17 \\
            &x_1 \geq 0, \text{nguyên} \\
            &x_2 \geq 0. \\
            \end{split}\right. \\
        \end{split}
        \end{equation*}
    \end{column}
    \begin{column}{0.4\textwidth}
        \begin{equation*}
            \Longrightarrow
            \left\{\begin{split}
            \color{red} x_1 &\color{red} = 1.3 \\
            \color{red} x_2 &\color{red} = 3.3 \\
            z&=14.08
        \end{split}\right.
        \end{equation*}
    \end{column}
    \begin{column}{0.1\textwidth}
    \end{column}
    \end{columns}
    \vspace{1cm}
    \center
    \Large
    Phương án có thể cải thiện.
\end{frame}


\begin{frame}{Ví dụ}
    \begin{columns}
    \begin{column}{0.1\textwidth}
    \end{column}
    \begin{column}{0.4\textwidth}
        \begin{equation*}
        \begin{split}
            (P) \quad & 3x_1 + 4x_2 \quad \longrightarrow Max \\
            & \left\{\begin{split}
            & 2.5x_1 + \frac{15}{4}x_2 \leq 20 \\
            & x_1 + \frac{5}{3}x_2 \leq \frac{50}{3} \\
            &x_1 \geq 0, \text{nguyên} \\
            &x_2 \geq 0. \\
            \end{split}\right. \\
        \end{split}
        \end{equation*}
    \end{column}
    \begin{column}{0.4\textwidth}
        \begin{equation*}
            \Longrightarrow
            \left\{\begin{split}
            \color{blue} x_1 &\color{blue} = 5 \\
            \color{blue} x_2 &\color{blue} = 7 \\
            z& =43
        \end{split}\right.
        \end{equation*}
    \end{column}
    \begin{column}{0.1\textwidth}
    \end{column}
    \end{columns}
    \vspace{1cm}
    \center
    \Large
    Bài toán được giải.
\end{frame}

\section{Thuật toán nhánh cận}
\begin{frame}
   \center
   \huge Thuật toán nhánh cận \\ (Branch-and-Bound)
\end{frame}

\begin{frame}
 
\end{frame}

%template to copy-paste
\begin{frame}

\end{frame}

\end{document}