\documentclass{beamer}
\usepackage[utf8]{vietnam}
\usepackage{graphicx}
\usepackage{booktabs}
\usetheme{Warsaw}
\title{TỐI ƯU TUYẾN TÍNH NGUYÊN}
\subtitle{PGS. TS Tạ Quang Sơn}
\author{Đỗ Ngọc Minh Thư \\ Nguyễn Chí Bằng}
\institute{Khoa Toán Ứng dụng \\ Trường Đại học Sài Gòn}
\date{\today}

\begin{document}

\begin{frame}
\titlepage
\end{frame}

\begin{frame}
\begin{center}
{\huge BUỔI XÉT DUYỆT ĐỀ TÀI}
\\[2\baselineskip]
\textbf{THÀNH VIÊN HỘI ĐỒNG}

TS. Nguyễn Văn Huấn

PGS. TS. Võ Hoàng Hưng

TS. Lương Duy Bình
\end{center}
\end{frame}


\begin{frame}
    \frametitle{NỘI DUNG BÁO CÁO}
    \tableofcontents
\end{frame}

\section{Mục đích nghiên cứu}

\begin{frame}{Mục đích nghiên cứu}
    Tối ưu tuyến tính là một nội dung quan trọng trong chương trình đào tạo Cử nhân Toán ứng dụng. Lý thuyết về việc giải bài toán tối ưu tuyến tính đã được cung cấp cho sinh viên. Tuy vậy, có nhiều bài toán tối ưu cần được giải với nghiệm nguyên. Chẳng hạn như:
    \begin{itemize}
    \item Bài toán tối ưu nhân lực vận chuyển hàng hóa.
    \item Bài toán tối ưu áp dụng trong tin học.
    \end{itemize}
\end{frame}

\section{Nội dung nghiên cứu}
\begin{frame}{Nội dung nghiên cứu}
    \begin{itemize}
    \item Hệ thống lại cơ sở lý thuyết và phương pháp giải các bài toán Quy hoạch tuyến tính.
    \item Từ đó mở rộng để tìm nghiệm nguyên cho bài toán, thông qua 2 phương pháp:
    \begin{itemize}
    \item Phương pháp Gomory.
    \item Phương pháp Land-Doig.
    \end{itemize}
    \end{itemize}
\end{frame}
\section{Dự kiến luận văn}
\begin{frame}{Dự kiến luận văn}
    \begin{itemize}
    \item Chương 1. Quy hoạch tuyến tính và cơ sở lý thuyết
    \begin{itemize}
        \item Các thuật toán đơn hình.
        \item Dạng đối ngẫu.
    \end{itemize}
    \item Chương 2. Quy hoạch nguyên
    \begin{itemize}
        \item Phương pháp Gomory.
        \item Phương pháp Land-Doig.
    \end{itemize}
    \item Chương 3. Vấn đề và ứng dụng của Quy hoạch nguyên
    \begin{itemize}
        \item Bài toán cái túi.
        \item Bài toán người du lịch.
        \item Bài toán gia công chi tiết máy. 
    \end{itemize}
    \end{itemize}   
\end{frame}
\section{Tổ chức và phân công}
\begin{frame}{Tổ chức và phân công}
    \begin{table}
        \begin{tabular}{|c|c|}
            \hline
            Họ và tên & Nội dung \\
            \hline \hline
            Đỗ Ngọc Minh Thư & Nghiên cứu về tối ưu tuyến tính nguyên. \\
            Nguyễn Chí Bằng & Nghiên cứu về tối ưu tuyến tính nguyên. \\
            \hline
        \end{tabular}
    \end{table}
\end{frame}
\section{Tiến độ thực hiện}
\begin{frame}[shrink=25]
    \frametitle{Tiến độ thực hiện}
    \vspace{2cm}
    \begin{table}
        \begin{tabular}{|c|c|c|}
            \hline
            Công việc thực hiện & Thời gian & Người thực hiện \\
            \hline \hline
            Nghiên cứu tối ưu tuyến tính nguyên. & 1/8/2023 - 1/4/2024 & Đỗ Ngọc Minh Thư \\
            Nghiên cứu tối ưu tuyến tính nguyên. & 1/8/2023 - 1/4/2024 & Nguyễn Chí Bằng \\
            \hline
        \end{tabular}
    \end{table}
\end{frame}
\section{Tài liệu dùng cho nghiên cứu}
\begin{frame}[allowframebreaks]
    \frametitle{Tài liệu dùng cho nghiên cứu}
    \bibliographystyle{amsalpha}
    \bibliography{tailieunc}
\end{frame}
\end{document}

